% Preamble
\documentclass[12pt]{article}

% Packages
\usepackage[a4paper, includefoot,
  left=3cm, right=1.5cm,
  top=2cm, bottom=2cm,
headsep=1cm, footskip=1cm]{geometry}
\usepackage{amsmath}
\usepackage{amssymb}
\usepackage[T2A]{fontenc}
\usepackage[utf8]{inputenc}
\usepackage[english, russian]{babel}
\usepackage{stmaryrd}
\usepackage{pdfpages}
\usepackage{graphicx}
\usepackage{wrapfig}
\usepackage{amsthm}
\usepackage{framed}
\usepackage{xcolor}
\usepackage{color}
\usepackage[unicode]{hyperref}

\hypersetup{
  colorlinks=true,
  linkcolor=black,
  urlcolor=blue,
}

\input{letters_series_mathbb}

\setcounter{tocdepth}{2}
\graphicspath{{../img}}

\theoremstyle{plain}
\newtheorem{statement}{Утверждение}
\newtheorem{theorem}{Теорема}

\theoremstyle{definition}
\newtheorem{definition}{Определение}
\newtheorem{property}{Свойство}
\newtheorem{example}{Пример}
\newtheorem*{corollary}{Следствие}

\theoremstyle{remark}
\newtheorem{remark}{Замечание}

\DeclareEmphSequence{\bfseries}

\begin{document}
Пусть элементы одномерного временного ряда $\tX$ длины $N$ имеют вид
\begin{equation*}
  x_n = \sum_{i=1}^{M} a_i e^{-\alpha_i n} \cos(2 \pi n \omega_i + \varphi_i),
\end{equation*}
где $a_i \in \bbR \setminus \{0\}$, $\alpha_i \in \bbR$, $\omega_i
\in (0, 1/2]$, $\varphi_i \in [0, 2\pi)$.
\begin{theorem}
  В условиях выше алгоритм ESPRIT способен точно оценить параметры
  $\omega_i$ при отсутствии шума тогда и только тогда, когда
  $\omega_i \ne \omega_j\
  \forall i\ne j$ и $\min(L, K) > \operatorname{rank}(\tX)$~\cite{?}.
\end{theorem}
\begin{remark}
  Условия предыдущей теоремы равносильны тому, что в пространстве
  подрядов $\tX$ длины $L$ существует базис из векторов вида
  \begin{gather*}
    \left\{\left(e^{-\alpha}\cos(2\pi n \omega)\right)_{n=1}^{L},\, \omega \in
    \Omega,\, \alpha \in A \right\}, \\
    \left\{\left(e^{-\alpha}\sin(2\pi n \omega)\right)_{n=1}^{L},\, \omega \in
    \Omega \setminus\{1/2\},\, \alpha \in A\right\},
  \end{gather*}
  где $\Omega$ и $A$ "--- множества всех частот $\omega_i$ и степеней
  затухания $\alpha_i$ соответственно, присутствующих в ряде.
\end{remark}

Рассмотрим преобразование исходного одномерного ряда $\tX$ в многомерный ряд
$\tX_D$, в котором $i$-й элемент ряда $\tX_D^{(d)}$ выбирается как
$x_n^{(d)} = x_{(n-1)D + d}$.
То есть ряд с номером $d$ строится выбором каждой $D$-й точки исходного ряда,
начиная с элемента $x_d$ и имеет длину $N_D = \lfloor N / D \rfloor$.
Это преобразование будем назвать Dstack.
Визуализация представлена на рисунке~\ref{fig:dstack-diagram}.
\begin{figure}[!ht]
  \centering
  \includegraphics[width=\textwidth]{dstack_diagram.pdf}
  \caption{Визуализация Dstack преобразования одномерного ряда в многомерный.}
  \label{fig:dstack-diagram}
\end{figure}

\begin{theorem}
  Рассмотрим алгоритмы ESPRIT и HO-ESPRIT для многомерных рядов с
  выбором длины окна $L$,
  применённые к результату преобразования Dstack одномерного ряда
  $\tX$ с параметром $D$.
  Тогда между частотами $\omega$ в исходном ряде и оценками
  $\widetilde{\omega}$ частот преобразованного ряда существует
  взаимно однозначное соответствие тогда и только тогда, когда
  \begin{enumerate}
    \item  $\omega_i \ne \omega_j\ \forall i\ne j$,
    \item  $\min(L, N_D - L + 1) > \operatorname{rank}(\tX)$
    \item
      \begin{equation*}
        \max_{\omega \in \Omega}\omega \leqslant \frac{1}{2D},
      \end{equation*}
  \end{enumerate}
  причём это соответствие задаётся выражением
  \begin{equation*}
    \omega = \frac{|\widetilde{\omega}|}{D}.
  \end{equation*}
\end{theorem}
\begin{proof}
  Достаточно доказать, что указанные условия равносильны тому, что в
  пространстве подрядов $\tX_D^{(d)}$ длины $L$
  существует базис из векторов вида
  \begin{gather*}
    \left\{\left(e^{-\alpha n}\cos(2\pi n
      D\omega)\right)_{n=1}^{L},\, \omega \in
    \Omega,\, \alpha \in \bbR \right\},\\
    \left\{\left(e^{-\alpha n }\sin(2\pi n
      D\omega)\right)_{n=1}^{L},\, \omega \in
    \Omega \setminus\{1/2\},\, \alpha \in \bbR\right\}.
  \end{gather*}

  Элементы ряда $\tX_D^{(d)}$ имеют вид
  \begin{align*}
    x_n^{(d)} &= \sum_{i=1}^{M} a_i e^{-\alpha_i ((n - 1)D + d)}
    \cos(2 \pi ((n - 1)D + d) \omega_i + \varphi_i) \\
    &=\sum_{i=1}^{M} a_i e^{-\alpha_i(d - D)} e^{-\alpha_i nD }
    \cos(2 \pi nD \omega_i + \omega_i (d - D) + \varphi_i).
  \end{align*}
  Тогда существование искомого базиса равносильно тому, что длина $L$
  каждого подряда больше размерности этого базиса, то есть больше
  $\operatorname{rank}(\tX)$, и количество $K = N_D - L + 1$ таких подрядов тоже
  больше $\operatorname{rank}(\tX)$.
\end{proof}
\begin{corollary}
  Из доказательства следует, что всегда существует взаимно
  однозначное соответствие между степенями затухания $\alpha$ исходного ряда и
  оценками степеней затухания $\widetilde{\alpha}$ методами ESPRIT и
  HO-ESPRIT после применения преобразования Dstack,
  и это соответствие выражается как $\alpha = \widetilde{\alpha} / D$.
\end{corollary}
\begin{corollary}
  Так как в пространствах подрядов каждого ряда $\tX_D^{(d)}$
  существует нужный базис, то в методе HO-ESPRIT выбор параметра $R_3
  < \operatorname{rank}_3(\calX_D)$, где $\calX$ "--- траекторный
  тензор $\tX_D$, не даёт смещения в оценке параметров.
  Это отличительная особенность задачи оценки параметров, в
  задаче выделения сигнала выбор $R_3 < \operatorname{rank}_3(\calX)$ даёт
  смещение всегда.
\end{corollary}
\end{document}