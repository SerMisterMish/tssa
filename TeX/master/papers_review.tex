% Preamble
\documentclass[12pt]{article}

% Packages
\usepackage[a4paper, includefoot,
  left=3cm, right=1.5cm,
  top=2cm, bottom=2cm,
headsep=1cm, footskip=1cm]{geometry}
\usepackage{amsmath}
\usepackage{amssymb}
\usepackage[T2A]{fontenc}
\usepackage[utf8]{inputenc}
\usepackage[english, russian]{babel}
\usepackage{pdfpages}
\usepackage{graphicx}
\usepackage{wrapfig}
\usepackage{amsthm}
\usepackage{framed}
\usepackage{xcolor}
\usepackage{color}
\usepackage[unicode]{hyperref}

\hypersetup{
  colorlinks=true,
  linkcolor=black,
  urlcolor=blue,
}

\include{letters_series_mathbb}

\setcounter{tocdepth}{2}
\graphicspath{{../img}}

\theoremstyle{plain}
\newtheorem{statement}{Утверждение}[section]
\newtheorem{theorem}{Теорема}

\theoremstyle{definition}
\newtheorem{definition}{Определение}[section]
\newtheorem{property}{Свойство}[section]
\newtheorem{example}{Пример}[section]
\newtheorem*{corollary}{Следствие}

\theoremstyle{remark}
\newtheorem{remark}{Замечание}[section]

\newcommand{\HOSVD}{\emph{HOSVD}}
\newcommand{\HOOI}{\emph{HOOI}}
\newcommand{\ESPRIT}{\emph{ESPRIT}}
\newcommand{\TSVD}{\emph{TSVD}}
\newcommand{\CPD}{\emph{CPD}}

\DeclareEmphSequence{\bfseries}

\begin{document}
\title{Обзор литературы по Tensor SSA}
\date{}
\author{}
\maketitle
\section{Статьи с теорией тензорных разложений}
\subsection{\href{https://doi.org/10.1137/s0895479896305696}
{A Multilinear Singular Value Decomposition}}\label{DeLathauwer2000}
Базовая теория
по \HOSVD{} (определения, свойства).

\subsection{\href{https://doi.org/10.1137/s0895479898346995}{On the Best
    Rank-1 and Rank-\texorpdfstring{$(R_1 ,R_2 ,. . .,R_N)$}{(R1, R2,
..., RN)} Approximation of Higher-Order Tensors}}\label{DeLathauwer2000a}
Про наилучшее приближение тензора
меньшими рангами, описание алгоритма \HOOI{}, некоторые его свойства.

\subsection{\href{https://doi.org/10.1109/TIT.2018.2841377}{Tensor SVD:
Statistical and Computational Limits}}
Рассматривается
точность приближения \HOOI{} произвольного тензора по его
зашумлённому варианту при различных случаях отношения минимального
сингулярного числа тензора к уровню шума (SNR).

\subsection{
  \href{https://doi.org/10.1016/j.laa.2010.09.020}{Factorization
  strategies for third-order tensors} и
  \href{https://doi.org/10.1137/110837711}{Third-Order
    Tensors as Operators on Matrices: A Theoretical and Compu\-tational
Framework with Applications in Imaging}}\label{Kilmer2011}
В основном теория по
трёхмерным тензорам,
вводится определение \TSVD{} и его свойства.

\section{Tensor SSA с использованием HOSVD или HOOI}
\subsection{\href{https://doi.org/10.1002/nla.453}{Exponential data
    fitting using multilinear algebra: the single-channel and
multi-channel case}}\label{Papy2005}
Рассматривается задача оценки параметров комплексного сигнала
(одномерный и многомерный случаи), состоящего из
суммы комплексных экспонент с близкими частотами.
Приводится описание и обоснование тензорной модификации алгоритма
\ESPRIT{} с применением \HOOI{}.
Проводится численное сравнение этой модификации с базовым \ESPRIT{}.
Выявлено преимущество тензорного метода по точности оценки параметров
сигнала, причём с увеличением уровня шума, преимущество увеличивается.

Траекторным тензором одномерного ряда $\tX$ с параметрами $I,L: 1 < I,L < N,\,
I + L < N + 1$ считается тензор $\mathcal{X}$ размера $I\times L \times J,\,
J=N-I-L+2$, элементы которого удовлетворяют равенству
\[
  \mathcal{X}_{ilj}=x_{i+l+j-2}\qquad i\in \overline{1:I},\, l
  \in\overline{1:L},\, j \in\overline{1:J}.
\]
Визуализация на рисунке~\ref{fig:traj-hosvd-ssa}.
\begin{figure}[!ht]
  \centering
  \includegraphics[width=\textwidth]{tens-injection-wide.pdf}
  \caption{Траекторный тензор одномерного ряда в
  HOSVD-SSA.}\label{fig:traj-hosvd-ssa}
\end{figure}

Траекторным тензором многомерного ряда $\tX$ с длиной окна $L:\: 1< L
< N$ считается тензор $\calX$ размерности ${L \times K \times P}$,
${K = N - L + 1}$, элементы которого удовлетворяют равенству
\[
  \calX_{lkp}=x_{l+k-1}^{(p)} \qquad l \in \overline{1:L},\, k \in
  \overline{1:K},\, p \in \overline{1:P}.
\]
Визуализация на рисунке~\ref{fig:traj-hosvd-mssa}.
\begin{figure}[!ht]
  \centering
  \includegraphics[width=\textwidth]{mssa_injection_new.pdf}
  \caption{Траекторный тензор многомерного ряда в HOSVD-MSSA.}
  \label{fig:traj-hosvd-mssa}
\end{figure}

Алгоритм заключается в нахождении наилучшего приближения траекторного
тензора ряда с $n$-рангами $(R, R, R)$ с помощью
алгоритма \HOOI{}.
$R$ задаётся равным числу экспонент с различными показателями входящих в сигнал.
Затем оценка строится по матрице сингулярных векторов одного из
направлений (1-го направления в одномерном случае, и 3-го в
многомерном) тем же способом, что и в базовом \ESPRIT{}.

\subsection{\href{https://doi.org/10.1002/cem.1212}{Exponential data
fitting using multilinear algebra: the decima\-tive case}}\label{Papy2009}
Рассматривается задача оценки параметров
одномерного комплексного сигнала (только одномерный случаи),
состоящего из суммы экспоненциально-модулированных гармоник с
близкими частотами.
Исследуется алгоритм HTLSDstack: модификация ESPRIT, в которой
по одномерному ряду строится $D$ прореженных рядов длины $M = N / D$ (считается,
что длина ряда $N$ делится на $D$ нацело).
Затем они считаются отдельными каналами одного многомерного ряда, и применяется
многомерный вариант ESPRIT.
Это уменьшает трудоёмкость алгоритма при небольшом уменьшении точности.
Предлагается тензорная модификация этого алгоритма: HO-HTLSDstack, в которой
к полученному многомерному ряду применяется тензорная модификация
многомерного \ESPRIT{} из~\ref{Papy2005}.

Исходный ряд $\tX$ длины $N$ разбивается на $D$ прореженных
подрядов $\tX^{(d)}$ длины $M$ так, что $x_m^{(d)}=x_{(m-1)D + d}$.
Другими словами, в ряд с номером $d$ входит каждый $D$-й элемент
исходного ряда, начиная с $x_d$.
По полученному многомерному ряду строится траекторный тензор так же,
как в~\ref{Papy2005}.
Затем ищется наилучшее приближение этого тензора с $n$-рангами
$(R,\min(R, M), R')$, где $R$ задаётся равным числу экспонент с
различными показателями входящих в сигнал, а $R' \leqslant \min(R, D)$.
Авторы утверждают, что если частоты гармоник близки, то выбор $R' < \min(R, D)$
улучшает точность оценки параметров.

Для оценки параметров используются сингулярные векторы 1-го
направления полученного разложения аппроксимирующего тензора.

Численно показано, что тензорный вариант (HO-HTLSDstack) с выбором
$R'=1$ оказывается точнее HTLSDstack во всех тестах, причём
преимущество увеличивается с увеличением уровня шума.
Также в одном тесте HO-HTLSDstack сравнивается с базовым ESPRIT и
так же оказывается более точной.
Во всех тестах рассматривался случай близких частот.

Кроме того, авторы показывают, что метод HO-HTLSDstack меньше чем на
порядок более трудоёмкий, чем HTLSDstack, но на порядок менее
трудоёмкий, чем базовый ESPRIT.

Графики со сравнением методов по точности и трудоёмкости, приведённые
в статье, указаны на рисунке~\ref{fig:papy2009}.
\begin{figure}[!ht]
  \centering
  \includegraphics[height=0.8\textheight]{papy2009_figures.png}
  \caption{Сравнение HTLS (ESPRIT), HTLSDstack и HO-HTLSDstack по
  точности оценки параметров и по трудоёмкости.}
  \label{fig:papy2009}
\end{figure}

\section{Tensor SSA с использованием \texorpdfstring{$(L_r, L_r,
1)$}{(Lr, Lr, 1)}-разложения}
\subsection{\href{https://doi.org/10.1137/100805510}{Blind Separation
    of Exponential Polynomials and the Decom\-position of a Tensor in
Rank-\texorpdfstring{$(L_r, L_r, 1)$}{(Lr, Lr, 1)} Terms}}
Приводится теоретическая информация про разложение в сумму тензоров с
$n$-рангами $(L_r, L_r, 1)$, в частности определение и условия единственности.

Рассматривается задача выделения сигнала в многомерном ряде
\[
  \bfY = \bfM \bfS + \bfN,
\]
где $\bfY$ "--- наблюдаемый ряд, $\bfS$ "--- искомый сигнал, $\bfM$
"--- коэффициенты линейных комбинаций, с которыми сигнал составляет
наблюдаемый ряд, $\bfN$ "--- шум.
Траекторный тензор ряда определяется так же, как в~\ref{Papy2005}.

Для модели, в которой сигнал составляют суммы произведений полиномов
и комплексных экспонент, доказаны условия единственности $(L_r, L_r,
1)$-разложения траекторного тензора ряда.

Метод заключается в построении траекторного тензора по ряду $\bfY$,
аппроксимации этого тензора меньшими $n$-рангами (но б\'{о}льшими, чем
$n$-ранги самого сигнала), и применении $(L_r, L_r, 1)$-разложения к
этой аппроксимации.
Далее по этому разложению можно построить оценку $\bfS$.

Проводятся численные сравнения точности выделения сигнала
предложенным методом при различных выборах параметров рангов
аппроксимации и $L_r$.
Сравнения с другими методами выделения сигнала не проводится.

\section{Tensor SSA с использованием CPD}
\subsection{\href{https://doi.org/10.1109/MLSP.2013.6661921}{Tensor based
singular spectrum analysis for nonstationary source separation}}
\end{document}