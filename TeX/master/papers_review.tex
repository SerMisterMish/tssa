% Preamble
\documentclass[12pt]{article}

% Packages
\usepackage[a4paper, includefoot,
  left=3cm, right=1.5cm,
  top=2cm, bottom=2cm,
headsep=1cm, footskip=1cm]{geometry}
\usepackage{amsmath}
\usepackage{amssymb}
\usepackage[T2A]{fontenc}
\usepackage[utf8]{inputenc}
\usepackage[english, russian]{babel}
\usepackage{pdfpages}
\usepackage{graphicx}
\usepackage{wrapfig}
\usepackage{amsthm}
\usepackage{framed}
\usepackage{xcolor}
\usepackage{color}
\usepackage[unicode]{hyperref}

\hypersetup{
  colorlinks=true,
  linkcolor=black,
  urlcolor=blue,
}

\include{letters_series_mathbb}

\setcounter{tocdepth}{2}
\graphicspath{{../img}}

\theoremstyle{plain}
\newtheorem{statement}{Утверждение}[section]
\newtheorem{theorem}{Теорема}

\theoremstyle{definition}
\newtheorem{definition}{Определение}[section]
\newtheorem{property}{Свойство}[section]
\newtheorem{example}{Пример}[section]
\newtheorem*{corollary}{Следствие}

\theoremstyle{remark}
\newtheorem{remark}{Замечание}[section]

\DeclareEmphSequence{\bfseries}

\begin{document}
\title{Обзор литературы по Tensor SSA}
\date{}
\author{}
\maketitle
\section{Статьи с теорией тензорных разложений}
\subsection{\href{https://doi.org/10.1137/s0895479896305696}
{A Multilinear Singular Value Decomposition}}\label{DeLathauwer2000}
Базовая теория
по \emph{HOSVD} (определения, свойства).

\subsection{\href{https://doi.org/10.1137/s0895479898346995}{On the Best
    Rank-1 and Rank-\texorpdfstring{$(R_1 ,R_2 ,. . .,R_N)$}{(R1, R2,
..., RN)}    Approximation of Higher-Order Tensors}}\label{DeLathauwer2000a}
Про наилучшее приближение тензора
меньшими рангами, описание алгоритма \emph{HOOI}, некоторые его свойства.

\subsection{\href{https://doi.org/10.1109/TIT.2018.2841377}{Tensor SVD:
Statistical and Computational Limits}}
Рассматривается
точность приближения \emph{HOOI} произвольного тензора по его
зашумлённому варианту при различных случаях отношения минимального
сингулярного числа тензора к уровню шума (SNR).

\subsection{
  \href{https://doi.org/10.1016/j.laa.2010.09.020}{Factorization
  strategies for third-order tensors} и
  \href{https://doi.org/10.1137/110837711}{Third-Order
    Tensors as Operators on Matrices: A Theoretical and Computational
Framework with Applications in Imaging}}\label{Kilmer2011}
В основном теория по
трёхмерным тензорам,
вводится определение \emph{TSVD} и его свойства.

\section{Tensor SSA с использованием HOSVD или HOOI}
\subsection{\href{https://doi.org/10.1002/nla.453}{Exponential data
    fitting using multilinear algebra: the single-channel and
multi-channel case}}\label{Papy2005}
Рассматривается задача оценки параметров
комплексного сигнала (одномерный и многомерный случаи), состоящего из
суммы экспоненциально-модулированных гармоник с близкими частотами.
Приводится описание и обоснование тензорной модификации алгоритма
\emph{ESPRIT} с применением \emph{HOOI}.
Проводится численное сравнение модификации с базовым \emph{ESPRIT}.
Выявлено преимущество тензорного метода по точности оценки параметров
сигнала, причём с увеличением уровня шума, преимущество увеличивается.

\subsection{\href{https://doi.org/10.1002/cem.1212}{Exponential data
fitting using multilinear algebra: the decimative case}}
Рассматривается задача оценки параметров
одномерного комплексного сигнала (только одномерный случаи),
состоящего из суммы экспоненциально-модулированных гармоник с
близкими частотами.
Исследуется алгоритм HTLSDstack: модификация ESPRIT, в которой
изначальный ряд делится на $D$ равных непересекающихся подрядов (считается,
что длина ряда $N$ делится на $D$ нацело), затем они считаются
отдельными каналами одного многомерного ряда, и применяется
многомерный вариант ESPRIT.
Это делается для уменьшения трудоёмкости алгоритма в угоду его точности.
Предлагается тензорная модификация этого алгоритма: HO-HTLSDstack, в которой
к полученному многомерному ряду применяется тензорная модификация
многомерного \emph{ESPRIT} из~\ref{Papy2005}.
\end{document}