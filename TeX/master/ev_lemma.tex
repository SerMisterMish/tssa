\documentclass{article}
\usepackage{cmap}
\usepackage[T2A]{fontenc}
\usepackage[utf8]{inputenc}
\usepackage[russian]{babel}
\usepackage{amsmath, amssymb, amsthm}
\usepackage{geometry}

\newtheorem{theorem}{Лемма}

\begin{document}
\begin{theorem}
  Рассмотрим матрицу \(A\in\mathbb{R}^{m\times n}\). Обозначим её
  столбцы \(a_1,\dots,a_n\). Пусть все столбцы \(A\) различны, матрица
  \(B\in\mathbb{R}^{m\times k}\, (k \geqslant n)\) составлена из
  столбцов \(A\) так,
  что каждый столбец из матрицы \(A\) появляется в \(B\), и некоторые
  столбцы могут повторяться. Пусть \(r_j\) (\(j=1,\dots,n\)) "---
  количество раз, которое столбец \(a_j\) входит в матрицу \(B\), и пусть
  \(M=\max_j r_j\). Тогда
  \[
    \sigma_i(B)\leqslant \sigma_i(A)\sqrt{M}\quad \forall i
  \]
  где \(\sigma_i(\cdot)\) обозначает \(i\)-е сингулярное значение
  матрицы в порядке убывания.
\end{theorem}

\begin{proof}
  Рассмотрим матрицу \(S\in\mathbb{R}^{n\times k}\) такую, что её
  \(k\)-й столбец "--- это вектор стандартного базиса
  \(e_j\in\mathbb{R}^n\) если \(k\)-й столбец матрицы \(B\) совпадает
  с \(a_j\). По построению \(B=AS\).
  Для любого \(j\) \(j\)-ая строка \(S\) содержит ровно \(r_j\) единиц.

  Обозначим столбцы матрицы \(B\) как \(b_t\).
  Рассмотрим \(x\in\mathbb{R}^k\),
  пусть \(y=Sx\in\mathbb{R}^n\). \(j\)-ая компонента
  \(y\) равна
  \[
    y_j=\sum_{t:\:b_t=a_j} x_t,
  \]
  по неравенству Коши--Буняковского,
  \[
    |y_j|\leqslant \sqrt{r_j}\left(\sum_{t:\:b_t=a_j}
    x_t^2\right)^{1/2}
    \leqslant \sqrt{M}\,\left(\sum_{t:\:b_t=a_j}
    x_t^2\right)^{1/2}.
  \]
  Следовательно
  \[
    \|y\|_2^2 = \sum_{j=1}^n |y_j|^2 \leqslant \sum_{j=1}^n M
    \left(\sum_{t:\: b_t=a_j}x_t^2\right) = M \sum_{t=1}^k x_t^2 = M \|x\|_2^2,
  \]
  так как множества \(\{t:\: b_t=a_j\}\) не пересекаются, и в
  объединении дают множество \(\{1,\dots,k\}\).
  Но \(\|y\|_2=\|Sx\|_2\), и значит \(\|S\|_{2}\leqslant \sqrt{M}\).

  Кроме того, для любого \(x\in\mathbb{R}^k\),
  \[
    \|Bx\|_2=\|A(Sx)\|_2 \le \|A\|_{2}\,\|Sx\|_2
    \le \|A\|_{2}\,\|S\|_{2}\,\|x\|_2
    \le \sqrt{M}\,\|A\|_{2}\,\|x\|_2.
  \]
  2-норма матрицы совпадает с её наибольшим сингулярным числом, поэтому
  \[
    \sigma_1(B)=\|B\|_{2}\leqslant
    \sqrt{M}\,\|A\|_{2}=\sqrt{M}\,\sigma_1(A).
  \]

  Пусть теперь \(i=1,\dots,n\), тогда по теореме Куранта--Фишера
  \[
    \sigma_i(B)=\min_{\substack{U\subset\mathbb{R}^k\\\dim U=k-i+1}}
    \;\max_{x\in U,\,x\neq\mathbf{0}} \frac{\|Bx\|_2}{\|x\|_2}.
  \]
  Зафиксируем произвольное подпространство \(U\subset\mathbb{R}^k\)
  размерности \(k-i+1\) и положим
  \[
    V=S(U) = \left\{v:\: v = Su,\, \forall u \in U\right\}\subset\mathbb{R}^n,
  \]
  тогда \(\dim V \leqslant \dim U = k - i + 1\), \(y = Sx \in V\).

  Пусть \(x\in U\) и \(x\neq\mathbf{0}\), тогда
  \[
    \frac{\|Bx\|_2}{\|x\|_2}=\frac{\|A(Sx)\|_2}{\|x\|_2}
    =\frac{\|A y\|_2}{\|x\|_2} = \frac{\|A
    y\|_2}{\|y\|_2}\frac{\|y\|_2}{\|x\|_2}=
    \frac{\|A y\|_2}{\|y\|_2}\frac{\|Sx\|_2}{\|x\|_2}
    \leqslant \frac{\|A y\|_2}{\|y\|_2}\frac{\|S\|_2 \|x\|_2}{\|x\|_2}
    \leqslant \sqrt{M} \frac{\|A y\|_2}{\|y\|_2},
  \]
  так как \(\|S\|_2\leqslant\sqrt{M}\).
  Переходя к максимуму по \(x\) и минимуму по \(U\), левая часть
  неравенства преобразуется в \(\sigma_i(B)\), а правая в
  \(\sqrt{M}\sigma_i(A)\).
\end{proof}

Применяя лемму, когда \(A\) "--- матрица с каналами сигнала в
строках, а \(B\) "--- 3-развёртка траекторного тензора, получаем, что
\[
  \sigma_i(B) \leqslant \sqrt{\min (L, K)}\,\sigma_i(A),
\]
где \(L\) "--- длина окна, \(K = N - L + 1\).

\end{document}