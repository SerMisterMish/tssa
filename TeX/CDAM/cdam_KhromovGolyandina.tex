%LaTeX SAMPLE FILE FOR PAPERS OF CDAM

% LaTeX 2e
\documentclass[12pt]{article}

% Delete when translated
\usepackage[T2A]{fontenc}

\usepackage{graphicx}
\usepackage{epsfig}
\usepackage{cite}
\usepackage{amsmath,amssymb,amsfonts,amsthm}
\usepackage[boxed]{algorithm2e}
\usepackage{caption}
\usepackage{wrapfig}
%\usepackage{subfig}
\usepackage{subcaption}
\usepackage{upgreek}
\usepackage{adjustbox}
\usepackage{pbox}
\usepackage{index}
\usepackage{color}
\usepackage{makecell}
\usepackage{multirow}
\usepackage{tabularx}
\usepackage{bbm}
\usepackage{lastpage}
\usepackage{longtable}
\usepackage{changepage}

% LaTeX 2.09
%\documentstyle[12pt]{article}

%%%%%%%%%%%%%%%%%%%%%%%%%%%% paper layout %%%%%%%%%%%%%%%%%%%%%%%%%%%%%%
\hoffset=-1in
\voffset=-1in
% Please, don't change this layout
\parindent=6mm
\topskip=0mm
\topmargin=30mm
\oddsidemargin=27.5mm
\evensidemargin=27.5mm
\textwidth=155mm
\textheight=237mm
\headheight=0pt
\headsep=0pt
\footskip=2\baselineskip
\addtolength{\textheight}{-\footskip}

\providecommand{\keywords}[1]
{
  \vspace{2mm}\hspace{20pt}\textbf{\textit{Keywords:}} #1
}

\providecommand{\abskeyw}[2]
{
  \begin{small}
    \begin{adjustwidth}{10mm}{10mm}
      \vspace{1mm}\hspace{20pt}#1

      \keywords{#2}
    \end{adjustwidth}
  \end{small}
}

\newcommand{\tX}{\mathsf{X}}
\newcommand{\bfX}{\mathbf{X}}
\newcommand{\calX}{\mathcal{X}}
\newcommand{\calT}{\mathcal{T}}
\newcommand{\calH}{\mathcal{H}}
\newcommand{\rmH}{\mathrm{H}}
\newcommand{\iu}{\mathrm{i}}

\theoremstyle{definition}
\newtheorem{definition}{Definition}

\theoremstyle{remark}
\newtheorem{remark}{Remark}

\begin{document}

%%% Title section
\begin{center}
  {\Large\bf TENSORS FOR SIGNAL AND FREQUENCY ESTIMATION IN
  SUBSPACE-BASED METHODS: WHEN THEY ARE USEFUL?}\\\vspace{2mm} {\sc N.A.
  Khromov$^1$, N.E. Golyandina$^2$}\\\vspace{2mm}
  {\it $^{1}$ $^{2}$St.\,Petersburg State University\\
  St.\,Petersburg, Russia\\} e-mail: {\tt $^1$hromovn@mail.ru,
  $^2$n.golyandina@spbu.ru}

  \abskeyw{В работе рассматриваются тензорные модификации singular
    spectrum analysis для решения задач выделения сигнала и
    оценки частот в зашумленной сумме экспоненциально-модулированных
    синусоид. Рассматриваются модификации с использованием Higher-Order
    SVD. Проводится численное сравнение. Численно показано, что для
    задачи выделения сигнала тензорные методы проигрывают
    матричным в большинстве случаев при одноканальном ряде, но могут
    выигрывать у многоканального SSA для системы рядов. Для оценки частот
  тензорные модификации, как правило, выигрывают.}{time series,
  signal, frequency estimation, tensor, singular spectrum analysis}
\end{center}

\section{Introduction}

Одним из методов анализа временных рядов является singular spectrum
analysis (SSA)~\cite{Golyandina2001}, в котором исходный временной ряда
трансформируется в матрицу, называемую траекторной, по заданной длине
окна $L$ и далее анализируется сингулярное разложение (SVD) этой
матрицы. Если стоит задача оценки сигнала и его свойств по
наблюдаемому зашумленному ряду, то рассматриваются первые $r$
компонент SVD, где $r$ --- ранг траекторной матрицы сигнала. На
основе выбранных компонент строится оценка сигнала. Отличительной чертой
метода является то, что он не требует задания
модели сигнала. Однако одновременно SSA позволяет работать с
параметрической моделью сигнала в виде суммы произведений полиномов
экспонент и синусоид. Особую роль играет оценка частот. На основе
оценки подпространства сигнала с  помощью первых $r$ левых
сингулярных векторов методом ESPRIT строится оценка частот,
присутствующих в сигнале.
LS версия ESPRIT~\cite{Roy1986} также называется HSVD,
а TLS версия~\cite{Roy1989} --- HTLS.
%Пусть сигнал задан в виде суммы синусоид (или комплексных экспонент
% в комплексном случае).
%Свойства оценок частот и оценок сигнала сильно различаются. В
% частности, дисперсия  оценки сигнала имеет порядок $1/N$, в то
% время как дисперсия оценок частот имеет порядок $1/N^3$, где $N$
% --- длина временного ряда и шум белый, гауссовский.

В ряде работ предлагаются тензорные модификации методов SSA и ESPRIT,
где исходно ряд трансформируется не в матрицу, а в тензор, как
правило, размерности три. Одним из распространенных вариантов
тензорных разложений является Higher-Order SVD (HO-SVD),
обобщающий матричный SVD.

Целью данной работы является численное сравнение тензорных и
матричных модификаций SSA для решения задач оценки сигнала и
оценки частот. Будем рассматривать тензорные модификации,
предлагаемые в работах~\cite{Papy2005} и~\cite{Papy2009}, расширенные
для выделения сигнала.

\section{Методы}
\subsection{Схема Tensor SSA для выделения сигнала}
Общая структура тензорных SSA алгоритмов на основе HO-SVD следующая
(обычный SSA является его частным случаем). Пусть $\tX$ - наблюдаемый
объект. В качестве длины окна рассматриваются размеры тензора по каждому
из трех направлений $I$, $L$ and $K$, где часть из них выражается
через другие или фиксируется. Параметрами метода являются три
значения $R_1$, $R_2$ и $R_3$, например, равные $r$, но не всегда.
\begin{enumerate}
  \item
    Вложение $\bfX = \calT(\tX)$ --- траекторный тензор.
  \item
    Разложение $\bfX =\sum_{i=1}^{I} \sum_{l=1}^{J} \sum_{k=1}^{K}
    \mathcal{Z}_{ilk} U^{(1)}_{i}
    \circ U^{(2)}_{l} \circ U^{(3)}_{k}$.
  \item
    Группировка $\hat\bfX =\sum_{i=1}^{R_1} \sum_{l=1}^{R_2} \sum_{k=1}^{R_3}
    \mathcal{Z}_{ilk} U^{(1)}_{i}
    \circ U^{(2)}_{l} \circ U^{(3)}_{k}$.
  \item
    Получение из $\hat\bfX$ оценки сигнала $\hat\tX$ на основе
    структуры траекторного тензора и операции, обратной к вложению.
\end{enumerate}

Далее будем рассматривать два варианта исходных объектов:
одноканальные и многоканальные временные ряды.

\subsection{Тензоры вложения}
Пусть $\tX = (x_1, x_2, \ldots, x_N)$ --- (одноканальный) временной ряд
длины $N$, $x_n \in
\mathbb{C}$.
\begin{definition}
  Оператором вложения одноканального временного ряда в тензор с
  длинами окна $I$ и $L$:
  ${1< I,L < N},\, {I + L < N + 1}$
  будем называть отображение $\calT_{I,L}$, переводящее ряд $\tX$ в
  тензор $\calX \in \mathbb{C}^{I\times L \times K}$
  (${K= N - I - L + 2}$)
  по правилу $\mathcal{X}_{ilk}=x_{i+l+k-2}$, где $i\in \overline{1:I},\, l
  \in\overline{1:L},\, k \in\overline{1:K}$.
\end{definition}

Пусть $\tX = (\tX^{(1)}, \tX^{(2)}, \ldots, \tX^{(P)})$ --- многоканальный
временной ряд, состоящий из $P$ одноканальных временных рядов, также
называемых каналами.
\begin{definition}
  Оператором вложения многоканального ряда в тензор с длиной окна $L$:
  ${1< L < N}$ будем называть отображение $\calT_{L}$, переводящее $P$-канальный
  ряд $\tX$ в тензор $\calX \in \mathbb{C}^{L\times K \times P}$
  (${K = N - L + 1}$)
  по правилу $x_{l+k-1}^{(p)}$, где $l \in \overline{1:L},\, k \in
  \overline{1:K},\, p \in \overline{1:P}$.
\end{definition}

\subsection{Методы для оценки параметров сигнала.}
Рассмотрим в общем случае $P$-канальный временной ряд (включая
$P=1$) с элементами
\begin{gather*}
  x_n^{(p)}= \sum_{r=1}^{R} a_r^{(p)} e^{\alpha_r n} e^{\iu\left(2\pi
  \omega_r n + \varphi_r^{(p)}\right)},
\end{gather*}
где параметрами модели являются амплитуды $a_j^{(p)} \in
\mathbb{C}\setminus\{0\}$, фазы ${\varphi_j^{(p)} \in [0, 2\pi)}$,
частоты $\omega_j\in [0, 1/2]$ и степени затухания $\alpha_j \in \mathbb{R}$.
Алгоритм HO-ESPRIT, оценивающий частоты и степени затухания ряда,
определяется следующим образом.
После шага разложения
строится матрица $\mathbf{U} = \mathbf{U}_d = \left[U_1^{(d)} :
U_2^{(d)}:\ldots : U_{R_d}^{(d)}\right]$ для некоторого $d\in \{1, 2,
3\}$, и решается уравнение
\begin{equation*}
  \mathbf{U}^{\uparrow}=\mathbf{U}_{\downarrow}\mathbf{Z}
\end{equation*}
относительно матрицы $\mathbf{Z}$, где запись $\mathbf{U}^{\uparrow}$ обозначает
матрицу $\mathbf{U}$ без первой строки, а $\mathbf{U}_{\downarrow}$
--- без последней.
$R$ наибольших собственных чисел матрицы $\mathbf{Z}$ считаются
оценками $\lambda_r = e^{\alpha_r + 2\pi\iu \omega_r}$, из которых
можно получить параметры $\alpha_r$ и $\omega_r$.
%Базовые алгоритмы ESPRIT, использующие траекторную матрицу и SVD,
%можно получить из HO-ESPRIT аналогично тому, как из HO-SSA и HO-MSSA
%можно получить базовые SSA и MSSA.

\subsection{Dstack модификация}
В работе~\cite{Papy2009} для ускорения работы метода предлагается
преобразование одноканального ряда в многоканальный перед применением
тензорной модификации: $x_m^{(d)} = x_{(m-1)D + d}$, где $m \in
\overline{1:(N/D)}$. В той работе это применяется только для
модификации \linebreak ESPRIT, называемой HTLSDstack, но мы будем применять
данное преобразование временного ряда и для оценки сигнала, метод
назовем SSADstack. Тензорные модификации строятся как для многоканального ряда.

\section{Сравнение тензорных методов с матричными}
Все численные сравнения были проведены для временных рядов в виде
суммы двух синусоид.

Для одноканальных временных рядов и задачи выделения сигнала было
проведено сравнение следующих методов:
SSA, HO-SSA, SSADstack, HO-SSADstack c $R_3=\max$ и HO-SSADstack c
$R_3=1$. Было получено, что в большинстве случаев метод SSA
существенно выигрывает по точности, а если проигрывает, то
незначительно и только в очень узком диапазоне параметров, что делает
это небольшое преимущество нереализуемым на практике.
Среди Dstack методов наиболее точными являются SSADstack и
HO-SSADstack c $R_3=\max$
с небольшим различием в точности.

Для одноканальных временных рядов и задачи оценки частот рассматривался
сигнал в виде двух синусоид с близкими частотами. Сравнивались методы
ESPRIT, HO-ESPRIT, HTLSDstack, HO-HTLSDstack c $R_3=\max$ и
HO-HTLSDstack c $R_3=1$. Было получено, что при низком уровне шума
ESPRIT работает точнее, однако при среднем и большом уровне шума
HO-ESPRIT становится точнее при оптимальном выборе параметров, а
HO-HTLSDstack c $R_3=1$ обыгрывает все методы.

Для многоканальных временных рядов было получено, что в случае, когда
ряды являются суммой синусоид с одинаковыми частотами, тензорные
модификации дают  более точный результат, как в задаче выделения
сигнала, так и в задаче оценивания частот.

\section{Conclusion}
Проведенное численное сравнение показало разный эффект от тензорной
HO-SVD модификации для временных рядов.
Для выделения сигнала для одномерных временных рядов матричный
вариант однозначно лучше.
Для многоканальных временных рядов с одинаковыми частотами в каналах
и для задачи оценки частот тензорный вариант может давать выигрыш в точности.

%% please make bibitems content in a style below !!!
%% papers with "free style" bibitems content will be rejected !!!

\begin{thebibliography}{10}
  \bibitem{Golyandina2001}
  Golyandina~N.E., Nekrutkin~V.V., Zhigljavsky~A.A.~(2001).
  {\sl Analysis of Time Series Structure}.
  Chapman and Hall/CRC: Boca Raton.

  \bibitem{Roy1986}
  Roy~R., Paulraj~A., Kailath~T.~(1986).
  ESPRIT-A subspace rotation approach to estimation of parameters of
  cisoids in noise.
  {\sl IEEE Transactions on Acoustics, Speech, and Signal Processing}.
  Vol.~{\bf 34}, Num.~{\bf 5},
  pp.~1340-1342.

  \bibitem{Roy1989}
  Roy~R., Kailath~T.~(1989).
  ESPRIT-estimation of signal parameters via rotational invariance techniques.
  {\sl IEEE Transactions on Acoustics, Speech, and Signal Processing}.
  Vol.~{\bf 37}, Num.~{\bf 7},
  pp.~984-995.

  \bibitem{Papy2005}
  Papy~J.M., De~Lathauwer~L., Van~Huffel~S.~(2005).
  Exponential data fitting using multilinear algebra: the
  single-channel and multi-channel case.
  {\sl Linear Algebra with Applications}. Vol.~{\bf 12}, Num.~{\bf 8},
  pp.~809-826.

  \bibitem{Papy2009}
  Papy~J.M., De~Lathauwer~L., Van~Huffel~S.~(2009).
  Exponential data fitting using multilinear algebra: the decimative case.
  {\sl Journal of Chemometrics}. Vol.~{\bf 23}, Num.~{\bf 7-8},
  pp.~341-351s.

  % \bibitem{paper}
  % Jacobs~P.A., Lewis~P.A.W.~(1983). Stationary Discrete
  % Autoregressive-Moving Average
  % Time Series Generated by Mixtures. {\sl Journal of Time Series
  % Analysis}. Vol.~{\bf 4}, Num.~{\bf 1},
  % pp.~19-36.

  % \bibitem{book}
  % Johnson~N.L., Kotz~S., Balakrishnan~N.~(1997). {\sl Discrete
  % Multivariate Distributions}. Wiley: New York.

  % \bibitem{web}
  % Worldometers.info [Electronic resource] Mode of access:
  % \texttt{https://www.worldometers.info/coronavirus.} Date of access:
  % 27.02.2022.

\end{thebibliography}

\end{document}
