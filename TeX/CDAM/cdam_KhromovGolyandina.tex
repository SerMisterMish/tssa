%LaTeX SAMPLE FILE FOR PAPERS OF CDAM

% LaTeX 2e
\documentclass[12pt]{article}

% Delete when translated
\usepackage[T2A]{fontenc}

\usepackage{graphicx}
\usepackage{epsfig}
\usepackage{cite}
\usepackage{amsmath,amssymb,amsfonts,amsthm}
\usepackage[boxed]{algorithm2e}
\usepackage{caption}
\usepackage{wrapfig}
%\usepackage{subfig}
\usepackage{subcaption}
\usepackage{upgreek}
\usepackage{adjustbox}
\usepackage{pbox}
\usepackage{index}
\usepackage{color}
\usepackage{makecell}
\usepackage{multirow}
\usepackage{tabularx}
\usepackage{bbm}
\usepackage{lastpage}
\usepackage{longtable}
\usepackage{changepage}

% LaTeX 2.09
%\documentstyle[12pt]{article}

%%%%%%%%%%%%%%%%%%%%%%%%%%%% paper layout %%%%%%%%%%%%%%%%%%%%%%%%%%%%%%
\hoffset=-1in
\voffset=-1in
% Please, don't change this layout
\parindent=6mm
\topskip=0mm
\topmargin=30mm
\oddsidemargin=27.5mm
\evensidemargin=27.5mm
\textwidth=155mm
\textheight=237mm
\headheight=0pt
\headsep=0pt
\footskip=2\baselineskip
\addtolength{\textheight}{-\footskip}

\providecommand{\keywords}[1]
{
  \vspace{2mm}\hspace{20pt}\textbf{\textit{Keywords:}} #1
}

\providecommand{\abskeyw}[2]
{
  \begin{small}
    \begin{adjustwidth}{10mm}{10mm}
      \vspace{1mm}\hspace{20pt}#1

      \keywords{#2}
    \end{adjustwidth}
  \end{small}
}

\newcommand{\tX}{\mathsf{X}}
\newcommand{\bfX}{\mathbf{X}}
\newcommand{\calX}{\mathcal{X}}
\newcommand{\calT}{\mathcal{T}}
\newcommand{\calH}{\mathcal{H}}
\newcommand{\rmH}{\mathrm{H}}
\newcommand{\iu}{\mathrm{i}}

\newcolumntype{Y}{>{\centering\arraybackslash}X}

\theoremstyle{definition}
\newtheorem{definition}{Definition}

\theoremstyle{remark}
\newtheorem{remark}{Remark}

\begin{document}

%%% Title section
\begin{center}
  {\Large\bf TENSORS FOR SIGNAL AND FREQUENCY ESTIMATION IN
  SUBSPACE-BASED METHODS: WHEN THEY ARE USEFUL?}\\\vspace{2mm} {\sc N.A.
  Khromov$^1$, N.E. Golyandina$^2$}\\\vspace{2mm}
{\it $^{1}$ $^{2}$St.\,Petersburg State University\\
St.\,Petersburg, Russia\\} e-mail: {\tt $^1$hromovn@mail.ru,
$^2$n.golyandina@spbu.ru}

  \abskeyw{В работе рассматриваются тензорные модификации singular spectrum analysis для решения двух задач, выделения сигнала и оценки частот в зашумленной сумме экспоненциально-модулированных синусоид.  Рассматриваются модификации с использованием High-Order SVD. Проводится численное сравнение. Численно показано, что для задачи выделения сигнала тензорные модификации проигрывают тензорным в большинстве случаев в одномерном случае, но могут выигрывать у многомерного SSA для системы рядов. Для оценки частот тензорные модификации, как правило, выигрывают.}{time series, signal, frequency estimation, tensor, singular spectrum analysis}
\end{center}

\section{Introduction}

Одним из методов анализа временных рядов является singular spectrum analysis (SSA) \cite{...}, в котором исходный временной ряда трансформируется в матрицу, называемую траекторной, по заданной длине окна $L$ и далее анализируется сингулярное разложение (SVD) этой матрицы. Если идет речь об оценке сигнала и его свойствах по наблюдаемому зашумленному ряду, то рассматриваются первые $r$ компонент SVD, где $r$ --- ранг траекторной матрицы сигнала. На основе выбранных компонент строится оценка сигнала, при этом отличительной чертой метода является то, что он не требует задания модели сигнала. Однако одновременно SSA позволяет работать с параметрической моделью сигнала в виде суммы произведений полиномов экспонент и синусоид. Особую роль играет оценка частот. На основе оценки подпространства сигнала с  помощью первых $r$ левых сингулярных векторов методом ESPRIT (HSVD для LS версии и HTLS для TLS) \cite{...} строится оценка частот, присутствующих в сигнале. Пусть сигнал задан в виде суммы синусоид (или комплексных экспонент в комплексном случае). Свойства оценок частот и оценок сигнала сильно различаются. В частности, дисперсия  оценки сигнала имеет порядок $1/N$, в то время как дисперсия оценок частот имеет порядок $1/N^3$, где $N$ --- длина временного ряда и шум белый, гауссовский.

В ряде работ предлагаются тензорные модификации методов SSA и ESPRIT, где исходно ряд трансформируется не в матрицу, а в тензор, как правило, размерности три. Одним из распространенных вариантов тензорных разложений является HO-SVD, являющийся обобщением разложения SVD.

Целью данной работы является численное сравнение тензорных и матричных модификаций SSA для решения двух задач, оценки сигнала и оценки частот. Будем рассматривать тензорные модификации, предлагаемые в работах ..., расширенные для выделения сигнала.  

\section{Методы}
\subsection{Схема Tensor SSA для выделения сигнала}
Общая структура тензорных SSA алгоритмов на основе HO-SVD следующая (обычный SSA является его частным случаем). Пусть $\tX$ - наблюдаемый объект. Вместо длины окна рассматриваются порядки тензора по каждой из трех размерностей $I$, $J$ and $K$, где часть из них выражается через другие или фиксируется. Параметрами метода являются три значения $R_1$, $R_2$ и $R_3$, например, равные $r$, но не всегда.
\begin{enumerate}
\item
Вложение $\bfX = \calT(\tX)$ - траекторный тензор.
\item
Разложение $\bfX =\sum_{l=1}^{I} \sum_{k=1}^{J} \sum_{p=1}^{K}
        \mathcal{Z}_{lkp} U^{(1)}_{l}
        \circ U^{(2)}_{k} \circ U^{(3)}_{p}$.
\item
Группировка $\hat\bfX =\sum_{l=1}^{R_1} \sum_{k=1}^{R_2} \sum_{p=1}^{R_3}
        \mathcal{Z}_{lkp} U^{(1)}_{l}
        \circ U^{(2)}_{k} \circ U^{(3)}_{p}$.
\item
Получение из $\hat\bfX$ оценки сигнала $\hat\tX$ на основе структуры траекторного тензора и операции, обратной к вложению.
\end{enumerate}

Далее будем рассматривать два варианта исходных объектов, одноканальные и многоканальные временные ряды. 

\subsection{Тензоры вложения}
Пусть $\tX = (x_1, x_2, \ldots, x_N)$ --- (одноканальный) временной ряд
длины $N$, $x_n \in
\mathbb{C}$.
\begin{definition}[Оператор вложения временного ряда в тензор]
  Оператором вложения временного ряда в тензор с длинами окна $I$ и $L$:
  ${1< I,L < N},\, {I + L < N + 1}$
  будем называть отображение $\calT_{I,L}$, переводящее ряд $\tX$ в
  тензор $\calX \in \mathbb{C}^{I\times L \times J}$ \linebreak
  (${J= N - I - L + 2}$)
  по правилу $\mathcal{X}_{ilj}=x_{i+l+j-2}$, где $i\in \overline{1:I},\, l
  \in\overline{1:L},\, j \in\overline{1:J}$.
\end{definition}

Пусть $\tX = (\tX^{(1)}, \tX^{(2)}, \ldots, \tX^{(P)})$ --- $P$-канальный
временной ряд, состоящий из $P$ одноканальных временных рядов, также
называемых каналами.
\begin{definition}[Оператор вложения многоканального ряда в тензор]
  Оператором вложения многоканального ряда в тензор с длиной окна $L$:
  ${1< L < N}$ будем называть отображение $\calT_{L}$, переводящее $P$-канальный
  ряд $\tX$ в тензор $\calX \in \mathbb{C}^{L\times K \times P}$ \linebreak
  (${K = N - L + 1}$)
  по правилу $x_{l+k-1}^{(p)}$, где $l \in \overline{1:L},\, k \in
  \overline{1:K},\, p \in \overline{1:P}$.
\end{definition}

%Визуализации применения вложения к одноканальному и многоканальному
%рядам представлены
%на Figures~\ref{fig:1d_injection} and~\ref{fig:pd-injection}.
%\begin{figure}[!ht]
%  \centering
%  \subfloat[][Результат применения оператора вложения к одноканальному
%  временному ряду.]{
%    \includegraphics[width=0.6\textwidth]{img/tens-injection-wide.pdf}
%  \label{fig:1d_injection}} \qquad
%
%  \subfloat[][Результат применения оператора вложения к многоканальному
%  временному ряду.]{
%    \includegraphics[width=0.6\textwidth]{img/mssa_injection_new.pdf}
%    \label{fig:pd-injection}
%  }
%  \caption{Визуализации результатов применения операторов вложения
%  рядов в тензоры.}
%\end{figure}

%\subsection{Методы для выделения сигнала из временных рядов}
%В алгоритме~\ref{alg:hossa} представлен метод HO-SSA для выделения сигнала из
%одноканального временного ряда.
%\begin{algorithm}[!ht]
%  \caption{HO-SSA for signal extraction.}\label{alg:hossa}
%  \KwData{$\tX$, $I,L: 1< I,L < N,\, I + L < N + 1$, $R_1 \in \overline{1:I}$,
%  $R_2 \in \overline{1:L}$,\\\hspace{36pt}$R_3 \in \overline{1:J}$.}
%  \KwResult{$\widetilde{\tX}$ --- оценка сигнала $\tX.$}
%  \begin{enumerate}
%    \item \textbf{Вложение:} построение $\calX=\calT_{I, L}(\tX)$
%      \label{algstep:ssa-inj}\;
%    \item \textbf{Разложение:} применение HOSVD или HOOI к $\calX$
%      \label{algstep:ssa-decomp}
%      \begin{equation*}
%        \widehat{\mathcal{X}}=\sum_{i=1}^{R_1} \sum_{l=1}^{R_2} \sum_{j=1}^{R_3}
%        \mathcal{Z}_{ilj} U^{(1)}_{i}
%        \circ U^{(2)}_{l} \circ U^{(3)}_{j};
%      \end{equation*}
%    \item \textbf{Восстановление:} усреднение тензора
%      $\widehat{\mathcal{X}}$ вдоль
%      плоскостей $i+l+j=\operatorname{const}$,
%      в результате чего получается оценка сигнала $\widehat{\tX}$.
%  \end{enumerate}
%\end{algorithm}
%\begin{remark}
%  Применение алгоритма~\ref{alg:hossa} с такими параметрами длин
%  окна, что размер одного любого направления тензора вложения равен
%  1, сводит алгоритм к базовому методу SSA, так как применение HOSVD или HOOI
%  к тензору с двумя направлениями (матрице) совпадает с применением SVD.
%\end{remark}
%
%В алгоритме~\ref{alg:homssa} представлен метод HO-MSSA для выделения сигнала из
%многоканального временного ряда.
%\begin{algorithm}[!ht]
%  \caption{HO-MSSA for signal extraction.}\label{alg:homssa}
%  \KwData{$\tX = \left(\tX^{(1)}, \ldots, \tX^{(P)}\right)^{\mathrm{T}}$,
%    $L: 1< L < N$, $R_1 \in \overline{1:L}$,
%  $R_2 \in \overline{1:K}$, $R_3 \in \overline{1:P}$ ($K = N-L+1$).}
%  \KwResult{$\widetilde{\tX} = (\widetilde{\tX}^{(1)},
%      \widetilde{\tX}^{(2)}, \ldots,
%  \widetilde{\tX}^{(Q)})$ --- оценка сигнала $\tX$.}
%  \begin{enumerate}
%    \item \textbf{Вложение:} построение $\calX=\calT_{L}(\tX)$
%      \label{algstep:mssa-inj}\;
%    \item \textbf{Разложение:} применение HOSVD или HOOI к $\calX$
%      \label{algstep:mssa-decomp}
%      \begin{equation*}
%        \widehat{\mathcal{X}}=\sum_{l=1}^{R_1} \sum_{k=1}^{R_2} \sum_{p=1}^{R_3}
%        \mathcal{Z}_{lkp} U^{(1)}_{l}
%        \circ U^{(2)}_{k} \circ U^{(3)}_{p};
%      \end{equation*}
%    \item \textbf{Восстановление:} сечения $\mathring{\calX}_{\cdot\cdot p}$
%      усредняются вдоль побочных диагоналей $l + k = \operatorname{const}$ для
%      получения оценок $\widetilde{\tX}^{(p)}$.
%  \end{enumerate}
%\end{algorithm}
%(\emph{Возможно можно сократить запись, если написать, что шаг 1 такой
%    же как и в одномерном
%    алгоритме, но оператор вложения другой, шаг 2 полностью совпадает,
%шаг 3 обратный шагу 1?}.)
%\begin{remark}
%  Применение алгоритма~\ref{alg:homssa} к одноканальному
%  ряду также даёт базовый алгоритм SSA.
%\end{remark}
%
%\begin{remark}
%  Если в алгоритме~\ref{alg:homssa} слои третьего направления
%  траекторного тензора соединить в матрицу по столбцам (получится
%  матрица, состоящая из $P$ блоков-матриц $L\times K$), применить к
%  ней SVD, построить приближение этой матрицы по первым $R$
%  компонентам разложения, и затем применить антидиагональное
%  усреднение к каждму блоку-матрице, то получится метод MSSA.
%\end{remark}

\subsection{Методы для оценки параметров сигнала.}
Рассмотрим в общем случае $P$-канальный временной ряд (возможно $P=1$) с элементами
\begin{gather*}
%  \tX = (\tX^{(1)}, \tX^{(2)}, \ldots, \tX^{(P)}),\\
%  \tX^{(p)} = (x_1^{(p)}, x_2^{(p)}, \ldots, x_N^{(p)}), \quad
%  p=\overline{1:P},\\
  x_n^{(p)}= \sum_{r=1}^{R} a_r^{(p)} e^{\alpha_r n} e^{\iu\left(2\pi
  \omega_r n + \varphi_r^{(p)}\right)},
\end{gather*}
где параметрами модели являются амплитуды $a_j^{(p)} \in
\mathbb{C}\setminus\{0\}$, фазы ${\varphi_j^{(p)} \in [0, 2\pi)}$,
частоты $\omega_j\in [0, 1/2]$ и степени затухания $\alpha_j \in \mathbb{R}$.
Алгоритм HO-ESPRIT, оценивающий частоты и степени затухания ряда,
определяется следующим образом.
После шага разложения 
строится матрица $\mathbf{U} = \mathbf{U}_d = \left[U_1^{(d)} :
U_2^{(d)}:\ldots : U_{R_d}^{(d)}\right]$ для некоторого $d\in \{1, 2,
3\}$, и решается уравнение
\begin{equation*}
  \mathbf{U}^{\uparrow}=\mathbf{U}_{\downarrow}\mathbf{Z}
\end{equation*}
относительно матрицы $\mathbf{Z}$, где запись $\mathbf{U}^{\uparrow}$ означает
матрицу $\mathbf{U}$ без первой строки, а $\mathbf{U}_{\downarrow}$
--- без последней.
$R$ наибольших собственных чисел матрицы $\mathbf{Z}$ считаются
оценками $\lambda_r = e^{\alpha_r + 2\pi\iu \omega_r}$, из которых
можно получить параметры $\alpha_r$ и $\omega_r$.
%Базовые алгоритмы ESPRIT, использующие траекторную матрицу и SVD,
%можно получить из HO-ESPRIT аналогично тому, как из HO-SSA и HO-MSSA
%можно получить базовые SSA и MSSA.


\subsection{Dstack модификация}
В работе \cite{...} для ускорения работы метода предлагается преобразование одномерного ряда в многомерный перед применением тензорной модификации: $x_m^{(d)} = x_{(m-1)D + d}$, где $m \in \overline{1:(N/D)}$. В \cite{...} это применяется только для модификации ESPRIT, называемой HTLSDstack, но мы будем применять данное преобразование временного ряда и для оценки сигнала, метод назовем SSADstack. Тензорные модификации строятся как для многоканального ряда.
%При большой длине ряда $N$ вычисление HOSVD (SVD) траекторного
%тензора (матрицы) может быть довольно трудоёмкой задачей.
%Одним из способов бороться с этой проблемой является модификация
%алгоритма EPSRIT: HTLSDstack (HTLS --- другое название ESPRIT).
%Метод HTLSDstack разработан для одноканальных временных рядов.
%(\emph{Возможно здесь вставить фразу, что он обобщается на
%многоканальные ряды, но такой случай мы рассматривать в работе не будем.})
%
%Метод заключается в том, чтобы по одноканальному временнмоу ряду
%$\tX = (x_1, x_2, \ldots, x_N)$ построить многоканальный ряд
%$\tX_D = (\tX^{(1)}, \tX^{(2)}, \ldots, \tX^{(D)})$,
%где $D$ - некоторый параметр (предполагается, что $N$ делится на $D$ нацело),
%а элементы рядов $\tX_D^{(d)}$ получаются из оригинального ряда by
%decimating the time series by factor D.
%Другими словами, $x_m^{(d)} = x_{(m-1)D + d}$, где $m \in \overline{1:(N/D)}$.
%Затем к полученному многоканальному ряду применяется многоканальный
%вариант метода HO-ESPRIT или ESPRIT.
%По Nyquist-Shannon sampling theorem, можно увеличивать sampling time
%interval $\Delta t$ в $D$ раз с сохранением всех частот в сигнале, пока
%сохраняется равенство $\max\limits_{r}\left|\omega_r\right| < 1 / (2
%D \Delta t)$.


\section{Сравнение тензорных методов с матричными}
Все численные сравнения были проведены для временных рядов в виде суммы синусоид.

Для одномерных временных рядов и задачи выделения сигнала было проведено сравнение следующих методов:
SSA, HO-SSA, SSADstack, HO-SSADstack c $R_3=\max$ и HO-SSADstack c $R_3=1$. Было получено, что в большинстве случаев метод SSA существенно выигрывает по точности, а если проигрывает, то незначительно и только в очень узком диапазоне параметров, что делает это небольшое преимущество нереализуемым на практике. 

Для одномерных временных рядов и задачи оценки частот рассматривался сигнал в виду двух синусоид с близкими частотами. Сравнивались методы ESPRIT, HO-ESPRIT, HTLSDstack, HO-HTLSDstack c $R_3=\max$ и HO-HTLSDstack c $R_3=1$. Было получено, что при низком уровне шума ESPRIT работает точнее, однако при среднем и большом уровне шума HO-ESPRIT становится точнее при оптимальном выборе параметров, а HO-HTLSDstack c $R_3=1$ обыгрывает все методы.

Для многоканальных временных рядов было получено, что в случае, когда ряды являются суммой синусоид с одинаковыми частотами, тензорные модификации дают  более точный результат, как в задаче выделения сигнала, так и в задаче оценивания частот.
%Сравнение для одномерных методов в таблицах~\ref{tab:1d-rec},~\ref{tab:1d-est}.
%\begin{table}[!ht]
%  \centering
%  \caption{Сравнение результатов для одномерных методов выделения сигнала.}
%  \label{tab:1d-rec}
%  \begin{tabularx}{\textwidth}{|p{32mm}|X|} \hline
%    Метод & Результат сравнения точности \\ \hline
%    SSA & Обычно наиболее точный, ошибка с увеличением шума растёт
%    медленно. \\ \hline
%    HO-SSA & В большинстве случаев значительно менее точный, чем SSA, однако
%    можно подобрать примеры, когда он слегка точнее SSA.\\ \hline
%    SSADstack & Точность ниже, чем у базового SSA. \\ \hline
%    HO-SSADstack ($R_3 = \max$) & Слегка точнее, чем SSADstack, но всё ещё
%    менее точен, чем SSA.  \\ \hline
%    HO-SSADstack ($R_3 = 1$)& При малом шуме точность сильно ниже,
%    чем у SSA за счёт большого сдвига, с увеличением шума становится точнее, чем
%    SSADstack и HO-SSADstack, но всё ещё менее точный, чем базовый SSA.\\ \hline
%  \end{tabularx}
%\end{table}
%\begin{table}[!ht]
%  \caption{Сравнение результатов для одномерных методов оценки параметров.}
%  \label{tab:1d-est}
%  \begin{tabularx}{\textwidth}{|p{32mm}|X|} \hline
%    HTLS & При малом уровне шума точнее, чем Dstack варианты.
%    С увеличением шума в случае, когда частоты сигнала близки, перестаёт
%    идентифицировать одну из компонент быстрее, чем Dstack методы. \\ \hline
%    HO-HTLS & При выборе оптимальных параметров оказывается точнее, чем
%    HTLS, однако различие точности невелико.\\ \hline
%    HTLSDstack & Точность ниже, чем у базового HTLS.\\ \hline
%    HO-HTLSDstack ($R_3 = \max$) & Немного точнее HTLS, но заметно менее точен,
%    чем при $R_3=1$.\\ \hline
%    HO-HTLSDstack ($R_3 = 1$) & При малом шуме немного менее точный, чем
%    HTLS, но точнее остальных методов.
%    Более устойчив к большому шуму при близких частотах. \\ \hline
%  \end{tabularx}
%\end{table}

%Сравнение результатов для многомерных методов:
%\begin{enumerate}
%  \item Выделение сигнала: HO-MSSA точнее базового MSSA при выборе
%    одинаковых длин окна. Чем сильнее ряды похожи друг на друга, тем
%    больше преимущество.
%  \item Оценка параметров: аналогично выделению сигнала.
%\end{enumerate}

\section{Conclusion}
Проведенное численное сравнение показало разный эффект от тензорной HO-SVD модификации для временных рядов.
Для выделения сигнала для одномерных временных рядов матричный вариант однозначно лучше.
Для многоканальных временных рядов с одинаковыми частотами в каналах и для задачи оценки частот тензорный вариант может давать выигрыш в точности.
%% please make bibitems content in a style below !!!
%% papers with "free style" bibitems content will be rejected !!!

\begin{thebibliography}{10}

  \bibitem{Papy2005}
  Papy~J.M., De~Lathauwer~L., Van~Huffel~S.~(2005).
  Exponential data fitting using multilinear algebra: the
  single-channel and multi-channel case.
  {\sl Linear Algebra with Applications}. Vol.~{\bf 12}, Num.~{\bf 8},
  pp.~809-826.

  \bibitem{Papy2009}
  Papy~J.M., De~Lathauwer~L., Van~Huffel~S.~(2009).
  Exponential data fitting using multilinear algebra: the decimative case.
  {\sl Journal of Chemometrics}. Vol.~{\bf 23}, Num.~{\bf 7-8},
  pp.~341-351s.

  \bibitem{paper}
  Jacobs~P.A., Lewis~P.A.W.~(1983). Stationary Discrete
  Autoregressive-Moving Average
  Time Series Generated by Mixtures. {\sl Journal of Time Series
  Analysis}. Vol.~{\bf 4}, Num.~{\bf 1},
  pp.~19-36.

  \bibitem{book}
  Johnson~N.L., Kotz~S., Balakrishnan~N.~(1997). {\sl Discrete
  Multivariate Distributions}. Wiley: New York.

  \bibitem{web}
  Worldometers.info [Electronic resource] Mode of access:
  \texttt{https://www.worldometers.info/coronavirus.} Date of access:
  27.02.2022.

\end{thebibliography}


\end{document}
