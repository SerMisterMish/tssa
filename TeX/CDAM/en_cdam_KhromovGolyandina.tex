%LaTeX SAMPLE FILE FOR PAPERS OF CDAM

% LaTeX 2e
\documentclass[12pt]{article}

\usepackage{graphicx}
\usepackage{epsfig}
\usepackage{cite}
\usepackage{amsmath,amssymb,amsfonts,amsthm}
\usepackage[boxed]{algorithm2e}
\usepackage{caption}
\usepackage{wrapfig}
\usepackage{subcaption}
\usepackage{upgreek}
\usepackage{adjustbox}
\usepackage{pbox}
\usepackage{index}
\usepackage{color}
\usepackage{makecell}
\usepackage{multirow}
\usepackage{tabularx}
\usepackage{bbm}
\usepackage{lastpage}
\usepackage{longtable}
\usepackage{changepage}

% LaTeX 2.09
%\documentstyle[12pt]{article}

%%%%%%%%%%%%%%%%%%%%%%%%%%%% paper layout %%%%%%%%%%%%%%%%%%%%%%%%%%%%%%
\hoffset=-1in
\voffset=-1in
% Please, don't change this layout
\parindent=6mm
\topskip=0mm
\topmargin=30mm
\oddsidemargin=27.5mm
\evensidemargin=27.5mm
\textwidth=155mm
\textheight=237mm
\headheight=0pt
\headsep=0pt
\footskip=2\baselineskip
\addtolength{\textheight}{-\footskip}

\providecommand{\keywords}[1]
{
  \vspace{2mm}\hspace{20pt}\textbf{\textit{Keywords:}} #1
}

\providecommand{\abskeyw}[2]
{
  \begin{small}
    \begin{adjustwidth}{10mm}{10mm}
      \vspace{1mm}\hspace{20pt}#1

      \keywords{#2}
    \end{adjustwidth}
  \end{small}
}

\newcommand{\tX}{\mathsf{X}}
\newcommand{\bfX}{\mathbf{X}}
\newcommand{\calX}{\mathcal{X}}
\newcommand{\calT}{\mathcal{T}}
\newcommand{\calH}{\mathcal{H}}
\newcommand{\rmH}{\mathrm{H}}
\newcommand{\iu}{\mathrm{i}}

\theoremstyle{definition}
\newtheorem{definition}{Definition}

\begin{document}

%%% Title section
\begin{center}
  {\Large\bf TENSORS FOR SIGNAL AND FREQUENCY ESTIMATION IN
  SUBSPACE-BASED METHODS: WHEN THEY ARE USEFUL?}\\\vspace{2mm} {\sc N.A.
  Khromov$^1$, N.E. Golyandina$^2$}\\\vspace{2mm}
  {\it $^{1}$ $^{2}$St.\,Petersburg State University\\
  St.\,Petersburg, Russia\\} e-mail: {\tt $^1$hromovn@mail.ru,
  $^2$n.golyandina@spbu.ru}

  \abskeyw{Tensor modifications of singular spectrum analysis for
    signal extraction and frequency estimation problems in a noisy sum
    of exponentially modulated sinusoids are reviewed. Modifications
    using Higher-Order SVD are considered. Numerical comparisons are
    carried out. It is shown numerically that for the signal extraction
    problem, tensor methods are inferior to matrix methods in most
    cases for a single-channel series, but can outperform multichannel
    SSA for a series system. For frequency estimation, tensor
  modifications are generally advantageous.}{time series,
  signal, frequency estimation, tensor, singular spectrum analysis}
\end{center}

\section{Introduction}

One of the methods for time series analysis is singular spectrum
analysis (SSA)~\cite{Golyandina2001}, in which the original time
series is transformed into a matrix, called trajectory matrix, by a
given window length $L$ and then the singular value decomposition
(SVD) of this matrix is analyzed. If the task is to estimate the
signal and its properties from the observed noisy series, the first
$r$ components of the SVD are considered, where $r$ is the rank of
the signal trajectory matrix. Based on the selected components, the
signal estimation is constructed. A distinctive feature of the method
is that it does not require the specification of a signal model.
However, at the same time SSA allows to handle a parametric model of
the signal in the form of a sum of products of polynomials, exponents
and sinusoids. A special role is played by the frequency estimation
problem. Based on the estimation of the signal subspace using the
first $r$ left singular vectors of the trajectory matrix SVD,
the ESPRIT method estimates the frequencies present in the signal.
The LS version of ESPRIT~\cite{Roy1986} is also called HSVD, and the TLS
version~\cite{Roy1989} is called HTLS.
%Пусть сигнал задан в виде суммы синусоид (или комплексных экспонент
% в комплексном случае).
%Свойства оценок частот и оценок сигнала сильно различаются. В
% частности, дисперсия  оценки сигнала имеет порядок $1/N$, в то
% время как дисперсия оценок частот имеет порядок $1/N^3$, где $N$
% --- длина временного ряда и шум белый, гауссовский.

A number of works propose tensor modifications of the SSA and ESPRIT
methods, where the original series is transformed into a tensor,
usually of 3rd order, instead of a matrix.  One of the common
variants of tensor decompositions is Higher-Order SVD (HO-SVD), which
generalizes the matrix SVD.

The purpose of this work is to numerically compare tensor and matrix
modifications of SSA for solving signal extraction and frequency
estimation problems. We will consider the tensor modifications
proposed in~\cite{Papy2005} and~\cite{Papy2009}, extended for signal extraction.

\section{Methods description}
\subsection{Tensor SSA algorithm layout for signal extraction}
The general structure of tensor SSA algorithms based on HO-SVD is as
follows (basic SSA is its special case). Let $\tX$ be the observed
object. The tensor dimensions $I$, $L$ and $K$ are considered as the
window length, where some of them are expressed through others or
fixed. The parameters of the algorithm are the values $R_1$, $R_2$ and
$R_3$. They are frequently chosen to be equal to $r$, for example,
but not always.
\begin{enumerate}
  \item
    Embedding $\bfX = \calT(\tX)$ --- trajectory tensor.
  \item
    Decomposition $\bfX =\sum_{i=1}^{I} \sum_{l=1}^{J} \sum_{k=1}^{K}
    \mathcal{Z}_{ilk} U^{(1)}_{i}
    \circ U^{(2)}_{l} \circ U^{(3)}_{k}$.
  \item
    Grouping $\hat\bfX =\sum_{i=1}^{R_1} \sum_{l=1}^{R_2} \sum_{k=1}^{R_3}
    \mathcal{Z}_{ilk} U^{(1)}_{i}
    \circ U^{(2)}_{l} \circ U^{(3)}_{k}$.
  \item
    Obtaining from $\hat\bfX$ the signal estimate $\hat\tX$ based on
    the structure of the trajectory tensor and the operation inverse
    to embedding.
\end{enumerate}

We will further consider two variants of input objects:
single-channel and multi-channel time series.

\subsection{Trajectory tensors}
Let $\tX = (x_1, x_2, \ldots, x_N)$ be a single-channel time series
of length $N$, $x_n \in \mathbb{C}$.
\begin{definition}
  The tensor embedding operator for a single-channel time series with
  window lengths $I$ and $L$ such that ${1< I,L < N},\, {I + L < N + 1}$
  is a mapping $\calT_{I,L}$ that translates the series
  $\tX$ into the tensor $\calX \in \mathbb{C}^{I\times L \times K}$
  as follows ${\mathcal{X}_{ilk}=x_{i+l+k-2}}$, where
  $i\in \overline{1:I},\, l \in\overline{1:L},\, k \in\overline{1:K}$,
  $K= N - I - L + 2$.
\end{definition}

Let $\tX = (\tX^{(1)}, \tX^{(2)}, \ldots, \tX^{(P)})$ be a multi-channel
time series consisting of $P$ single-channel series, also called channels.
\begin{definition}
  The tensor embedding operator for a multi-channel time series with
  window length $L$ such that ${1< L < N}$ is a mapping $\calT_{L}$
  that translates the
  $P$-channel time-series $\tX$ into the tensor $\calX \in
  \mathbb{C}^{L\times K \times P}$ (${K = N - L + 1}$) as follows
  $\mathcal{X}_{lkp} =  x_{l+k-1}^{(p)}$, where ${l \in
  \overline{1:L}},\, {k \in \overline{1:K}},\, {p \in \overline{1:P}}$.
\end{definition}

\subsection{Algorithm for signal parameters estimation.}
Consider the $P$-channel time series (including $P=1$) with elements
\begin{gather*}
  x_n^{(p)}= \sum_{r=1}^{R} a_r^{(p)} e^{\alpha_r n} e^{\iu\left(2\pi
  \omega_r n + \varphi_r^{(p)}\right)},
\end{gather*}
where the model parameters are the amplitudes $a_r^{(p)} \in
\mathbb{C}\setminus\{0\}$, phases ${\varphi_r^{(p)} \in [0, 2\pi)}$,
frequencies $\omega_r\in [0, 1/2]$, and damping factors $\alpha_r \in
\mathbb{R}$. The HO-ESPRIT algorithm that estimates the frequencies
and damping factors of a time series is defined as follows.
After the embedding step the matrix $\mathbf{U} = \mathbf{U}_d =
\left[U_1^{(d)} :
U_2^{(d)}:\ldots : U_{R_d}^{(d)}\right]$ for $d\in \{1, 2,
3\}$ is constructed and the following matrix equation
\begin{equation*}
  \mathbf{U}^{\uparrow}=\mathbf{U}_{\downarrow}\mathbf{Z}
\end{equation*}
is solved with respect to matrix $\mathbf{Z}$, where the up and down
arrows placed
behind the matrix $\mathbf{U}$ stand for deleting its first and last
rows accordingly.
The $R$ largest eigenvalues of the matrix $\mathbf{Z}$ are considered
to be the estimates of poles $\lambda_r = e^{\alpha_r + 2\pi\iu
\omega_r}$, from which the parameters $\alpha_r$ and $\omega_r$ can be obtained.
%Базовые алгоритмы ESPRIT, использующие траекторную матрицу и SVD,
%можно получить из HO-ESPRIT аналогично тому, как из HO-SSA и HO-MSSA
%можно получить базовые SSA и MSSA.

\subsection{Dstack modifications}
In the paper~\cite{Papy2009}, to improve the speed of the method, it
is proposed to transform a single-channel series into a multi-channel
series before applying the tensor modification: $x_m^{(d)} =
x_{(m-1)D + d}$, where $m \in \overline{1:(N/D)}$. In that paper only
the ESPRIT modification called HTLSDstack is considered, but we will
apply this time series transformation for the signal estimation
problem as well, and will call the resulting method SSADstack.
Tensor modifications are constructed as for a multichannel series.

\section{Comparison of tensor methods with matrix methods}
All numerical comparisons were made for time series expressed as a
sum of sinusoids.

The following methods were compared for single channel time series
and signal extraction problem: SSA, HO-SSA, SSADstack, HO-SSADstack
with $R_3=r$ and HO-SSADstack with $R_3=1$. It was shown that in
most cases the SSA method significantly outperforms other methods in
terms of accuracy, and when it is less accurate, the difference is
insignificant and is present only in a very narrow range of
parameters, which makes this small advantage unrealizable in actual
practice. Among Dstack methods the most accurate are SSADstack and
HO-SSADstack with $R_3=r$ with small difference in accuracy.

For single-channel time series and frequency estimation problem, a
signal in the form of two sinusoids with close frequencies was
considered. The ESPRIT, HO-ESPRIT, HTLSDstack, HO-HTLSDstack with
$R_3=r$ and HO-HTLSDstack with $R_3=1$ methods were compared. It was
obtained that at low noise level ESPRIT performs more accurately, but
at medium and high noise level HO-ESPRIT becomes more accurate with
optimal parameter selection, and HO-HTLSDstack with $R_3=1$
outperforms all methods.

For multi-channel time series, it was shown that in the case when
all the channels of the series are expressed as a sum of sinusoids, and
the frequencies presented in each channel are equal, tensor
modifications give more accurate results, both in the signal
extraction problem and in the frequency estimation problem.

\section{Conclusion}
The performed numerical comparisons showed different effects of the
tensor HO-SVD modifications for different time series. For signal
extraction of a single-channel time series, the basic matrix method is
unambiguously more accurate. For multi-channel time series with equal
set of frequencies in the channels and for the frequency estimation
problem, the tensor methods can give an advantage in accuracy.

%% please make bibitems content in a style below !!!
%% papers with "free style" bibitems content will be rejected !!!

\begin{thebibliography}{10}
  \bibitem{Golyandina2001}
  Golyandina~N.E., Nekrutkin~V.V., Zhigljavsky~A.A.~(2001).
  {\sl Analysis of Time Series Structure}.
  Chapman and Hall/CRC: Boca Raton.

  \bibitem{Roy1986}
  Roy~R., Paulraj~A., Kailath~T.~(1986).
  ESPRIT-A subspace rotation approach to estimation of parameters of
  cisoids in noise.
  {\sl IEEE Transactions on Acoustics, Speech, and Signal Processing}.
  Vol.~{\bf 34}, Num.~{\bf 5},
  pp.~1340-1342.

  \bibitem{Roy1989}
  Roy~R., Kailath~T.~(1989).
  ESPRIT-estimation of signal parameters via rotational invariance techniques.
  {\sl IEEE Transactions on Acoustics, Speech, and Signal Processing}.
  Vol.~{\bf 37}, Num.~{\bf 7},
  pp.~984-995.

  \bibitem{Papy2005}
  Papy~J.M., De~Lathauwer~L., Van~Huffel~S.~(2005).
  Exponential data fitting using multilinear algebra: the
  single-channel and multi-channel case.
  {\sl Linear Algebra with Applications}. Vol.~{\bf 12}, Num.~{\bf 8},
  pp.~809-826.

  \bibitem{Papy2009}
  Papy~J.M., De~Lathauwer~L., Van~Huffel~S.~(2009).
  Exponential data fitting using multilinear algebra: the decimative case.
  {\sl Journal of Chemometrics}. Vol.~{\bf 23}, Num.~{\bf 7-8},
  pp.~341-351s.

  % \bibitem{paper}
  % Jacobs~P.A., Lewis~P.A.W.~(1983). Stationary Discrete
  % Autoregressive-Moving Average
  % Time Series Generated by Mixtures. {\sl Journal of Time Series
  % Analysis}. Vol.~{\bf 4}, Num.~{\bf 1},
  % pp.~19-36.

  % \bibitem{book}
  % Johnson~N.L., Kotz~S., Balakrishnan~N.~(1997). {\sl Discrete
  % Multivariate Distributions}. Wiley: New York.

  % \bibitem{web}
  % Worldometers.info [Electronic resource] Mode of access:
  % \texttt{https://www.worldometers.info/coronavirus.} Date of access:
  % 27.02.2022.

\end{thebibliography}

\end{document}
