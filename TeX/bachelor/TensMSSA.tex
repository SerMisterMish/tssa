% Preamble
\documentclass[11pt]{article}

% Packages
\usepackage{amsmath}
\usepackage{amsthm}
\usepackage{amssymb}
\usepackage[T2A]{fontenc}
\usepackage[utf8]{inputenc}
\usepackage[english, russian]{babel}
\usepackage{graphicx}
\usepackage{slashbox}
\usepackage{hhline}
\usepackage{multirow}

%new calligraphic font for subspaces 
\usepackage{euscript}
\newcommand{\cA}{\EuScript{A}}
\newcommand{\cB}{\EuScript{B}}
\newcommand{\cC}{\EuScript{C}}
\newcommand{\cD}{\EuScript{D}}
\newcommand{\cE}{\EuScript{E}}
\newcommand{\cF}{\EuScript{F}}
\newcommand{\cG}{\EuScript{G}}
\newcommand{\cH}{\EuScript{H}}
\newcommand{\cI}{\EuScript{I}}
\newcommand{\cJ}{\EuScript{J}}
\newcommand{\cK}{\EuScript{K}}
\newcommand{\cL}{\EuScript{L}}
\newcommand{\cM}{\EuScript{M}}
\newcommand{\cN}{\EuScript{N}}
\newcommand{\cO}{\EuScript{O}}
\newcommand{\cP}{\EuScript{P}}
\newcommand{\cQ}{\EuScript{Q}}
\newcommand{\cR}{\EuScript{R}}
\newcommand{\cS}{\EuScript{S}}
\newcommand{\cT}{\EuScript{T}}
\newcommand{\cU}{\EuScript{U}}
\newcommand{\cV}{\EuScript{V}}
\newcommand{\cW}{\EuScript{W}}
\newcommand{\cX}{\EuScript{X}}
\newcommand{\cY}{\EuScript{Y}}
\newcommand{\cZ}{\EuScript{Z}}

%font for text indices like transposition X^\mathrm{T}
\newcommand{\rmA}{\mathrm{A}}
\newcommand{\rmB}{\mathrm{B}}
\newcommand{\rmC}{\mathrm{C}}
\newcommand{\rmD}{\mathrm{D}}
\newcommand{\rmE}{\mathrm{E}}
\newcommand{\rmF}{\mathrm{F}}
\newcommand{\rmG}{\mathrm{G}}
\newcommand{\rmH}{\mathrm{H}}
\newcommand{\rmI}{\mathrm{I}}
\newcommand{\rmJ}{\mathrm{J}}
\newcommand{\rmK}{\mathrm{K}}
\newcommand{\rmL}{\mathrm{L}}
\newcommand{\rmM}{\mathrm{M}}
\newcommand{\rmN}{\mathrm{N}}
\newcommand{\rmO}{\mathrm{O}}
\newcommand{\rmP}{\mathrm{P}}
\newcommand{\rmQ}{\mathrm{Q}}
\newcommand{\rmR}{\mathrm{R}}
\newcommand{\rmS}{\mathrm{S}}
\newcommand{\rmT}{\mathrm{T}}
\newcommand{\rmU}{\mathrm{U}}
\newcommand{\rmV}{\mathrm{V}}
\newcommand{\rmW}{\mathrm{W}}
\newcommand{\rmX}{\mathrm{X}}
\newcommand{\rmY}{\mathrm{Y}}
\newcommand{\rmZ}{\mathrm{Z}}

%tt font for time series
\newcommand{\tA}{\mathsf{A}}
\newcommand{\tB}{\mathsf{B}}
\newcommand{\tC}{\mathsf{C}}
\newcommand{\tD}{\mathsf{D}}
\newcommand{\tE}{\mathsf{E}}
\newcommand{\tF}{\mathsf{F}}
\newcommand{\tG}{\mathsf{G}}
\newcommand{\tH}{\mathsf{H}}
\newcommand{\tI}{\mathsf{I}}
\newcommand{\tJ}{\mathsf{J}}
\newcommand{\tK}{\mathsf{K}}
\newcommand{\tL}{\mathsf{L}}
\newcommand{\tM}{\mathsf{M}}
\newcommand{\tN}{\mathsf{N}}
\newcommand{\tO}{\mathsf{O}}
\newcommand{\tP}{\mathsf{P}}
\newcommand{\tQ}{\mathsf{Q}}
\newcommand{\tR}{\mathsf{R}}
\newcommand{\tS}{\mathsf{S}}
\newcommand{\tT}{\mathsf{T}}
\newcommand{\tU}{\mathsf{U}}
\newcommand{\tV}{\mathsf{V}}
\newcommand{\tW}{\mathsf{W}}
\newcommand{\tX}{\mathsf{X}}
\newcommand{\tY}{\mathsf{Y}}
\newcommand{\tZ}{\mathsf{Z}}

%bf font for matrices
\newcommand{\bfA}{\mathbf{A}}
\newcommand{\bfB}{\mathbf{B}}
\newcommand{\bfC}{\mathbf{C}}
\newcommand{\bfD}{\mathbf{D}}
\newcommand{\bfE}{\mathbf{E}}
\newcommand{\bfF}{\mathbf{F}}
\newcommand{\bfG}{\mathbf{G}}
\newcommand{\bfH}{\mathbf{H}}
\newcommand{\bfI}{\mathbf{I}}
\newcommand{\bfJ}{\mathbf{J}}
\newcommand{\bfK}{\mathbf{K}}
\newcommand{\bfL}{\mathbf{L}}
\newcommand{\bfM}{\mathbf{M}}
\newcommand{\bfN}{\mathbf{N}}
\newcommand{\bfO}{\mathbf{O}}
\newcommand{\bfP}{\mathbf{P}}
\newcommand{\bfQ}{\mathbf{Q}}
\newcommand{\bfR}{\mathbf{R}}
\newcommand{\bfS}{\mathbf{S}}
\newcommand{\bfT}{\mathbf{T}}
\newcommand{\bfU}{\mathbf{U}}
\newcommand{\bfV}{\mathbf{V}}
\newcommand{\bfW}{\mathbf{W}}
\newcommand{\bfX}{\mathbf{X}}
\newcommand{\bfY}{\mathbf{Y}}
\newcommand{\bfZ}{\mathbf{Z}}

%bb font for standard spaces and expectation
\newcommand{\bbA}{\mathbb{A}}
\newcommand{\bbB}{\mathbb{B}}
\newcommand{\bbC}{\mathbb{C}}
\newcommand{\bbD}{\mathbb{D}}
\newcommand{\bbE}{\mathbb{E}}
\newcommand{\bbF}{\mathbb{F}}
\newcommand{\bbG}{\mathbb{G}}
\newcommand{\bbH}{\mathbb{H}}
\newcommand{\bbI}{\mathbb{I}}
\newcommand{\bbJ}{\mathbb{J}}
\newcommand{\bbK}{\mathbb{K}}
\newcommand{\bbL}{\mathbb{L}}
\newcommand{\bbM}{\mathbb{M}}
\newcommand{\bbN}{\mathbb{N}}
\newcommand{\bbO}{\mathbb{O}}
\newcommand{\bbP}{\mathbb{P}}
\newcommand{\bbQ}{\mathbb{Q}}
\newcommand{\bbR}{\mathbb{R}}
\newcommand{\bbS}{\mathbb{S}}
\newcommand{\bbT}{\mathbb{T}}
\newcommand{\bbU}{\mathbb{U}}
\newcommand{\bbV}{\mathbb{V}}
\newcommand{\bbW}{\mathbb{W}}
\newcommand{\bbX}{\mathbb{X}}
\newcommand{\bbY}{\mathbb{Y}}
\newcommand{\bbZ}{\mathbb{Z}}

%got font for any case
\newcommand{\gA}{\mathfrak{A}}
\newcommand{\gB}{\mathfrak{B}}
\newcommand{\gC}{\mathfrak{C}}
\newcommand{\gD}{\mathfrak{D}}
\newcommand{\gE}{\mathfrak{E}}
\newcommand{\gF}{\mathfrak{F}}
\newcommand{\gG}{\mathfrak{G}}
\newcommand{\gH}{\mathfrak{H}}
\newcommand{\gI}{\mathfrak{I}}
\newcommand{\gJ}{\mathfrak{J}}
\newcommand{\gK}{\mathfrak{K}}
\newcommand{\gL}{\mathfrak{L}}
\newcommand{\gM}{\mathfrak{M}}
\newcommand{\gN}{\mathfrak{N}}
\newcommand{\gO}{\mathfrak{O}}
\newcommand{\gP}{\mathfrak{P}}
\newcommand{\gQ}{\mathfrak{Q}}
\newcommand{\gR}{\mathfrak{R}}
\newcommand{\gS}{\mathfrak{S}}
\newcommand{\gT}{\mathfrak{T}}
\newcommand{\gU}{\mathfrak{U}}
\newcommand{\gV}{\mathfrak{V}}
\newcommand{\gW}{\mathfrak{W}}
\newcommand{\gX}{\mathfrak{X}}
\newcommand{\gY}{\mathfrak{Y}}
\newcommand{\gZ}{\mathfrak{Z}}

%old calligraphic font
\newcommand{\calA}{\mathcal{A}}
\newcommand{\calB}{\mathcal{B}}
\newcommand{\calC}{\mathcal{C}}
\newcommand{\calD}{\mathcal{D}}
\newcommand{\calE}{\mathcal{E}}
\newcommand{\calF}{\mathcal{F}}
\newcommand{\calG}{\mathcal{G}}
\newcommand{\calH}{\mathcal{H}}
\newcommand{\calI}{\mathcal{I}}
\newcommand{\calJ}{\mathcal{J}}
\newcommand{\calK}{\mathcal{K}}
\newcommand{\calL}{\mathcal{L}}
\newcommand{\calM}{\mathcal{M}}
\newcommand{\calN}{\mathcal{N}}
\newcommand{\calO}{\mathcal{O}}
\newcommand{\calP}{\mathcal{P}}
\newcommand{\calQ}{\mathcal{Q}}
\newcommand{\calR}{\mathcal{R}}
\newcommand{\calS}{\mathcal{S}}
\newcommand{\calT}{\mathcal{T}}
\newcommand{\calU}{\mathcal{U}}
\newcommand{\calV}{\mathcal{V}}
\newcommand{\calW}{\mathcal{W}}
\newcommand{\calX}{\mathcal{X}}
\newcommand{\calY}{\mathcal{Y}}
\newcommand{\calZ}{\mathcal{Z}}


\theoremstyle{plain}
\newtheorem{statement}{Утверждение}
\newtheorem{theorem}{Теорема}
\newtheorem{corollary}{Следствие}[statement]

\theoremstyle{definition}
\newtheorem{definition}{Определение}
\newtheorem{example}{Пример}[section]

\theoremstyle{remark}
\newtheorem*{remark}{Замечание}

% Document
\begin{document}

    \section{Описание метода Tensor MSSA}\label{sec:Tensor-MSSA-method-description}
    Пусть дан многомерный временной ряд $\tX$ длины $N$ размерности $P$, то есть
    \begin{gather*}
        \tX = (\tX^{(1)}, \tX^{(2)}, \ldots, \tX^{(P)})^{\rmT},\\
        \tX^{(p)} = (x_{p1}, x_{p2}, \ldots, x_{pN}).
    \end{gather*}
    Рассматривается задача выделения сигнала из ряда с шумом.

    \begin{definition}(Траекторный тензор многомерного ряда)
        Траекторным тензором многомерного ряда $\tX$ с параметром $L: 1\leqslant L \leqslant N$
        будем называть тензор $\mathcal{X}$ размерности $L\times K \times P$, где $K = N - L + 1$,
        слои вдоль третьего измерения которого удовлетворяют равенству
        \[
            \mathcal{X}_{,,p}=
            \begin{pmatrix}
                x_{p1} & x_{p2} & \ldots & x_{pK} \\
                x_{p2} & \ddots &        & \vdots \\
                \vdots &        & \ddots & \vdots \\
                x_{pL} & \ldots & \ldots & x_{pN}
            \end{pmatrix},
            \qquad p\in \overline{1:P}.
        \]
    \end{definition}

    Алгоритм Tensor MSSA для выделения сигнала из ряда с шумом сводится к получению
    как можно более точного приближения траекторного тензора тензором меньших $n$-рангов, задаваемых пользователем.

    \subsection{Ранг ряда в терминах Tensor MSSA}\label{subsec:tensor-mssa-rank}
    Следующее утверждение позволяет определить принцип выбора $n$-рангов в алгоритме Tensor MSSA.
    \begin{statement}
        Пусть $X_1, X_2, \ldots, X_P$ - временные ряды длины $N$.
        Пусть $\mathbf{H}_1, \ldots, \mathbf{H}_P$ - траекторные матрицы этих рядов с параметром $L \leqslant N$.
        Обозначим $K = N - L + 1$.

        Построим матрицу $\mathbf{X} = [\mathbf{H}_1: \mathbf{H}_2: \ldots: \mathbf{H}_P] \in
        \mathbb{R}^{L\times KP}$.
        Её \emph{SVD} имеет вид:
        \begin{gather}
            \mathbf{X} =\mathbf{U} \mathbf{\Lambda} \mathbf{V}.\label{mat-svd}
        \end{gather}

        Построим тензор $\mathcal{X} \in \mathbb{R}^{L\times K \times P}: \mathcal{X}_{,,p} = \mathbf{H}_p$
        ($p$-й слой тензора по 3-му измерению берём равным $\mathbf{H}_p$).
        Его \emph{HOSVD} имеет вид:
        \begin{gather}
            \mathcal{X}=\mathcal{Z} \times_1 \hat{\mathbf{U}}_1 \times_2 \hat{\mathbf{U}}_2 \times_3 \hat{\mathbf{U}}_3.\label{tens-hosvd}
        \end{gather}

        Существуют такие \emph{SVD} матрицы $\mathbf{X}$ и \emph{HOSVD} тензора $\mathcal{X}$, что
        $\mathbf{U} = \hat{\mathbf{U}}_1$.
    \end{statement}
    \begin{proof}
        Из свойств HOSVD известно, что в качестве $\hat{\mathbf{U}}_1$ можно выбрать матрицу $\hat{\mathbf{U}}$
        левых сингулярных векторов из SVD матрицы $[\mathcal{X}]_{(1)}$ -
        развёртки тензора $\mathcal{A}$ по первому измерению.
        Эта развёртка имеет вид
        \[
            [\mathcal{X}]_{(1)} = [\mathbf{H}_1: \mathbf{H}_2: \ldots: \mathbf{H}_P] = \mathbf{A},
        \]
        откуда и следует искомое утверждение.
    \end{proof}

    \begin{corollary}
        Из этого утверждения и из того, что разложения \emph{SVD} и \emph{HOSVD} единственны с точностью до, возможно,
        некоторых ортогональных преобразований матриц сингулярных векторов, следует,
        что пространства, порождаемые левыми сингулярными векторами матрицы $\mathbf{X}$
        и сингулярными векторами первого измерения тензора $\mathcal{X}$, совпадают.
    \end{corollary}
    \begin{remark}
        На практике при вычислении HOSVD тензора используется алгоритм, который последовательно
        вычисляет SVD развёрток этого тензора по каждому измерению, и получившиеся матрицы левых сингулярных векторов
        используются в качестве матриц сингулярных векторов соответствующего измерения в формуле~\eqref{tens-hosvd}.
        Таким образом, на практике совпадают не только пространства, порождаемые описанными выше сингулярными векторами,
        но и сами матрицы этих векторов.
    \end{remark}

    \begin{corollary}
        Пусть многомерный ряд $\tX$ длины $N$ имеет ранг $r$ в терминах \emph{MSSA}, и число $L \in \overline{1:N}$
        таково, что $r \leqslant \min(L, K)$, где $K = N - L + 1$, и пусть $\calX$ "--- траекторный тензор этого ряда.
        Тогда
        \[
            \operatorname{rank}_1(\calX) = r, \qquad \operatorname{rank}_2(\calX) = r.
        \]
    \end{corollary}
    \begin{remark}
        Ранг третьего измерения $\operatorname{rank}_3(\calX)$ имеет смысл отличный от смысла ранга ряда
        в теории MSSA.
        Этот ранг имеет смысл меры структурного отличия рядов друг от друга.
    \end{remark}

    \begin{example}[Ранги рядов]
        Пусть многомерный ряд $\tX$ имеет следующий вид:
        \begin{gather*}
            s_n^{(m)}=a_m \cos(2\pi n \omega_m + \psi_m),\, m \in \{1,\, 2\},\, n\in \overline{1:N},\\
            N > 7,\, a_m\ne 0,\, 0<\omega_m<\frac 1 2,\, 0 \leqslant \psi < 2 \pi.
        \end{gather*}
        Обозначим $r$ "--- ранг ряда $\tX$ в терминах MSSA,
        $r_i = \operatorname{rank}_i(\calX)$, где $\calX$ "--- траекторный тензор, построенный
        по ряду $\tX$ с длиной окна $L$ такой, что $r \leqslant \min(L, K),\, K = N - L + 1$.
        Тогда
        \begin{enumerate}
            \item если $\psi_1 = \psi_2,\, \omega_1 = \omega_2$, то $r = r_1 = r_2 = 2,\, r_3 = 1$,
            \item если $\psi_1 \ne \psi_2,\, \omega_1 = \omega_2$, то $r = r_1 = r_2 = 2,\, r_3 = 2$,
            \item если $\omega_1 \ne \omega_2$, то $r = r_1 = r_2 = 4,\, r_3 = 2$.
        \end{enumerate}
    \end{example}

    \subsection{Алгоритм HOSVD MSSA для выделения сигнала}\label{subsec:tensor-mssa-description}
    На вход алгоритму подаётся многомерный временной ряд $\tX$ длины $N$ и размерности $P$.
    Параметры алгоритма: $L:\: 1\leqslant L \leqslant N,\, R,\, R_3:\: R \leqslant \min(L, K),\, R_3 \leqslant P$,
    где $K = N - L + 1$.
    Алгоритм Tensor MSSA для выделения сигнала из ряда с шумом заключается в проведении следующих четырёх шагов.
    \begin{enumerate}
        \item Выбор параметра $L$ и построение по нему траекторного тензора $\mathcal{X}$;
        \item Проведение HOSVD траекторного тензора $\mathcal{X}$, получение его представления в виде
        \begin{equation}
            \mathcal{X}=\sum_{l=1}^{L} \sum_{k=1}^{K} \sum_{p=1}^{P} \mathcal{Z}_{l,k,p} \mathbf{U}^{(1)}_{l}
            \circ \mathbf{U}^{(2)}_{k} \circ \mathbf{U}^{(3)}_{p};
            \label{eq:trajectory-hosvd}
        \end{equation}
        \item Группировка: выбор параметров $R,\, R_3$ имеющих смысл числа компонент, относимых к сигналу,
        и построение тензора
        \begin{equation*}
            \hat{\mathcal{X}}=\sum_{l=1}^R \sum_{k=1}^R \sum_{p=1}^{R_3}
            \mathcal{Z}_{l,k,p} \mathbf{U}^{(1)}_{l}\circ \mathbf{U}^{(2)}_{k} \circ \mathbf{U}^{(3)}_{p}.
%            \label{eq:tens-m-group}
        \end{equation*}
        \item Восстановление сигнала $\hat{\tX}$ по тензору $\hat{\mathcal{X}}$ посредством его усреднения вдоль
        плоскостей $l+k+p=\operatorname{const}$:
        \begin{gather*}
            \hat{x}_n=\frac{1}{\#\mathfrak{M}_n}\sum_{(l,k,p)\in \mathfrak{M}_n} \hat{\mathcal{X}}_{l,k,p},
            \qquad n\in \overline{1:N},\\
            \mathfrak{M}_n=\{(l,\, k,\, p) \:|\: 1\leqslant l \leqslant L,\, 1\leqslant k \leqslant K,\, 1\leqslant p \leqslant P,\, l+k+p-2=n\}.
        \end{gather*}
    \end{enumerate}
    Результатом алгоритма является временной ряд $\hat{\tX}$, принимаемый за оценку сигнала.

    \begin{remark}
        В качестве параметра $R$ в общем случае рекомендуется выбирать ранг искомого сигнала, а в качестве
        параметра $R_3$ "--- ранг траекторного тензора сигнала по третьему измерению.
    \end{remark}

    \subsection{Сравнение HOSVD MSSA и MSSA}\label{subsec:mssa-comparison}
    Пусть сигнал задан многомерным временным рядом
    \begin{gather*}
        \tS =
        \begin{pmatrix}
            s_1^{(1)}, & s_2^{(1)}, & \ldots, s_N^{(1)} \\
            s_1^{(2)}, & s_2^{(2)}, & \ldots, s_N^{(2)}
        \end{pmatrix}\\
        s_n^{(m)} = a_m \cos(2\pi n \omega_m + \psi_m),\, N = 71,\, a_1 = 30,\, a_2 = 20.
    \end{gather*}
    На сигнал подействовали белым гауссовским шумом, с параметром ${\sigma=5}$.
     \begin{table}[!ht]
            \centering
            \caption{RMSE восстановленных с помощью MSSA и HOSVD MSSA сигналов для каждого набора параметров
            сигнала.}
            \def\arraystretch{1.7}
            \resizebox{\columnwidth}{!}{
            \begin{tabular}{r|r|rrrrr}
                \hline
                Условия & \backslashbox{Метод}{$L$}  & 12   & 24   & 36   & 48           & 60   \\ \hline
                \multirow{2}{*}{
                \begin{tabular}{r}
                           $\omega_1 = \omega_2 = \frac 1 {12}$\\
                           $\psi_1 = \psi_2 = 0$\\
                \end{tabular}
                } & MSSA & 1.78 & 1.34 & 1.24 & \textbf{1.20} & 1.42 \\ \cline{2-7}
                  & HOSVD MSSA & 1.35 & \textbf{1.10} & \textbf{1.10} & \textbf{1.10} & 1.35 \\ \hhline{=|=|=====}
                \multirow{2}{*}{
                \begin{tabular}{r}
                    $\omega_1 = \omega_2 = \frac 1 {12}$\\
                    $\psi_1 = 0,\, \psi_2 = \frac \pi  4$\\
                \end{tabular}
                } & MSSA & 1.78 & 1.34 & 1.25 & \textbf{1.20} & 1.41 \\ \cline{2-7}
                  & HOSVD MSSA & 1.41 & \textbf{1.19} & 1.20 & \textbf{1.19} & 1.41 \\ \hhline{=|=|=====}
                \multirow{2}{*}{
                \begin{tabular}{r}
                    $\omega_1 = \frac 1 {12},\, \omega_2 = \frac 1 8$\\
                    $\psi_1 = 0,\, \psi_2 = \frac \pi  4$\\
                \end{tabular}
                } & MSSA & 2.63 & 1.94 & 1.74 & \textbf{1.69} & 1.95 \\ \cline{2-7}
                  & HOSVD MSSA & 1.95 & \textbf{1.67} & 1.69 & \textbf{1.67} & 1.95 \\ \hline
            \end{tabular}
            }\label{tab:tens-mssa-comparison}
     \end{table}

    В таблице~\ref{tab:tens-mssa-comparison} приведены значения отклонения восстановленного ряда от исходного
    ряда для различных значений параметров после использования алгоритмов MSSA и HOSVD MSSA для выделения сигнала\@.
    RMSE посчитан по 500 реализациям шума, методы сравнивались на одних и тех же наборах реализаций шума.

\end{document}